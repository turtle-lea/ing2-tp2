\documentclass[a4paper]{article}
\usepackage[spanish]{babel}
\usepackage[utf8]{inputenc}
\usepackage{graphicx}
% \usepackage{pdfpages}
\usepackage{enumerate}
\usepackage{listings}
\usepackage{color}
\usepackage{indentfirst}
\usepackage{fancyhdr}
\usepackage{latexsym}
\usepackage[colorlinks=true, linkcolor=black]{hyperref}
\usepackage{wrapfig}
\usepackage{algpseudocode}
\usepackage{calc}
\usepackage{amsmath, amsthm, amssymb}
\usepackage{amsfonts}
\usepackage{lscape}
\usepackage{float}
\usepackage{hyperref}
\definecolor{gray}{gray}{0.5}
\definecolor{light-gray}{gray}{0.95}
\definecolor{orange}{rgb}{1,0.5,0}

\input{page.layout}
\usepackage{underscore}
\usepackage{caratula}
\usepackage{url}

\newcommand{\cod}[1]{{\tt #1}}
\newcommand{\negro}[1]{{\bf #1}}
\newcommand{\ital}[1]{{\em #1}}
\newcommand{\may}[1]{{\sc #1}}
\newcommand{\tab}{\hspace*{2em}}

\newcommand{\sprintstory}[6]{\begin{tabular}{| p{3cm} | p{12cm} |}
 \hline
 TargetProcess ID: & #1 \\
 \hline
 User Story: & #2 \\
 \hline
 Esfuerzo estimado: & #3 \\
 \hline
 Business Value: & #4 \\
 \hline
 Descripción: & #5 \\
 \hline
 Criterios de\newline Aceptación: & #6 \\
 \hline
\end{tabular}}

\newcommand{\usecase}[3]{\noindent\textbf{CU\##1. #2}\\
#3\\
~\\
}

\newenvironment{taskstable}
{ \begin{tabular}{| p{14cm} | p{1cm} |}
 \hline
 \multicolumn{2}{|c|}{{\bf División en tareas}}\\
 \hline
 {\bf Tarea} & {\bf HH}\\
 \hline }
{ \end{tabular} }

\newcommand{\task}[2]{#1 & #2\\
 \hline}

\hypersetup{
 pdfstartview= {FitH \hypercalcbp{\paperheight-\topmargin-1in-\headheight}},
 pdfauthor={Grupo},
 pdfsubject={Dise\~{n}o}
}

\lstset{escapechar=@}

\begin{document}

\thispagestyle{empty}
\materia{Ingeniería de Software II}
\submateria{Primer Cuatrimestre de 2016}
\titulo{Trabajo Práctico II: The Curry Game release v7.1.2}

\integrante{Leandro Matayoshi}{79/11}{leandro.matayoshi@gmail.com}
\integrante{Matías Pizzagalli}{257/12}{matipizza@gmail.com}
\integrante{Gastón Requeni}{400/11}{grequeni@hotmail.com}
\integrante{Martín Santos}{413/11}{martin.n.santos@gmail.com}

\makeatletter

\maketitle

\newenvironment{myindentpar}[1]
{\begin{list}{1}
         {\setlength{\leftmargin}{#1}}
         \item[]
}
{\end{list}}

\newcommand{\escenario}[8] {
  \noindent \underline {{#1}} \newline
  \noindent \textit{'{#2}'}
  \begin{itemize}
    \item \textbf{Fuente:} {#3}
    \item \textbf{Estímulo:} {#4}
    \item \textbf{Entorno:} {#5}
    \item \textbf{Artefacto:} {#6}
    \item \textbf{Respuesta:} {#7}
    \item \textbf{Medición de respuesta:} {#8}
  \end{itemize}
}

\newpage
% \section{Casos de uso}
% A continuación presentamos el diagrama de casos de uso simplificado (sólo mostramos relaciones entre casos de uso y actores, sin especificar dependencias).

\begin{figure}[h!]
  \centering
  \includegraphics[width=\textwidth]{imagenes/casosDeUso.png}
  \caption{Diagrama de Casos de Uso}
\end{figure}

\newpage

\subsection{Descripción de los casos de uso}

\subsubsection{Cuenta de usuario}
\usecase{01}
{Registrándose en la aplicación. (Participante)}
{Registro de participantes en la aplicación donde les solicitamos sus datos. Esto es necesario para poder autenticarse luego y poder usar la aplicación.}

\usecase{02}
{Autenticándose en la aplicación. (Participante, Administrador)}
{Único medio de ingreso a la aplicación para participar en desafíos o para realizar tareas administrativas (en el caso del Administrador).}

\usecase{03}
{Accediendo a estado de cuenta de participante. (Participante, Administrador)}
{Consulta/modificación de datos personales, dinero ganado, dinero perdido, estadísticas de desafíos ganados, equipos armados, etc. Un Administrador puede acceder con los mismos privilegios que el Participante a cuentas de participantes (como help desk o para extraer datos útiles).}

\subsubsection{Desafíos modo Simulación}

\usecase{04}
{Creando un desafío modo simulación. (Participante, Administrador)}
{Elección del tipo de torneo (plaoffs, liga, combinado zonas, etc), duración, deporte, cantidad máxima de participantes, fecha de inicio. Según la región del participante y según tenga acceso a niveles superiores, podrá elegir una región o se tomará la región del participante para mostrar el desafío a otros participantes de la misma región. Una vez creado empieza la cuenta regresiva para su comienzo. Usa el armado de equipo (CU\#10). Se indica premio si corresponde.}

\usecase{05}
{Participando en desafío modo simulación. (Participante)}
{Elección de un desafío de un listado de desafíos de la región. Usa el pago de la cuota de entrada (CU\#16). Usa también el armado de equipo (CU\#10). Al finalizar el desafío, ganará premios si corresponde.}

\usecase{06}
{Mirando partido simulado en tiempo real. (Participante)}
{Motor de simulación según deporte. Generación de un stream que se traduce a video o se reproduce usando el motor 3D o 2D según se determine en tiempo de ejecución. En cuanto al simulador de basquet, habrá que extenderlo según nuevos requerimientos. El resto de los motores se crean desde cero.}

\subsubsection{Desafíos modo Liga de Fantasía}

\usecase{07}
{Creando desafío modo liga de fantasía. (Participante, Administrador)}
{Elección de la cantidad de fechas reales que abarca, deporte, cantidad máxima de participantes, fecha de inicio. Según la región del participante y según tenga acceso a niveles superiores, podrá elegir una región o se tomará la región del participante para mostrar el desafío a otros participantes de la misma región. Una vez creado empieza la cuenta regresiva para su comienzo. Usa el armado de equipo (CU\#10).}

\usecase{08}
{Participando en desafío modo liga de fantasía. (Participante)}
{Elección de un desafío de un listado de desafíos de la región. Usa el pago de la cuota de entrada (CU\#16). Usa también el armado de equipo (CU\#10). Al finalizar el desafío, ganará premios si corresponde. Los resultados de cada fecha se calculan en base a las estadísticas reales.}

\usecase{09}
{Mirando en vivo partidos de ligas reales. (Participante)}
{Transmisión provista por dueños de los derechos de televisación en vivo de partidos reales.}


\subsubsection{Desafíos (simulados o liga de fantasía)}

\usecase{10}
{Armando equipo. (Participante)}
{Elección de jugadores en cada posición de la cancha y de los suplentes (difiere según deporte del desafío). Elección de un técnico para tener una estrategia de jugada o para sumar puntos según estadísticas (en el caso de liga de fantasía).}

\usecase{11}
{Consultado estado actual de un desafío. (Participante)}
{Estado del partido actual tomado de la simulación o tomado de las estadísticas en tiempo real del partido real. Estado del torneo en general, usando resultados de las simulaciones o los resultados de fechas anteriores de la liga de fantasía. Incluye ranking de participantes.}

\subsubsection{Chat}

\usecase{12}
{Chateando con participantes del desafío. (Participante)}
{Servicio de mensajería que permita intercambiar mensajes entre participantes de un mismo desafío.}

\usecase{13}
{Enviando IM a otro participante. (Participante)}
{Servicio de mensajería instantánea entre dos participantes.}

\subsubsection{Pagos y Cobros}

\usecase{14}
{Ingresando datos de tarjeta de crédito. (Participante)}
{La primera vez que se paga una cuota de entrada a desafío se solicitan los datos de tarjeta de crédito o cuenta corriente bancaria. En el caso de tarjeta de crédito, los datos se almacenan de forma segura y siempre pueden ser modificados.}

\usecase{15}
{Ingresando datos de cuenta corriente. (Participante)}
{La primera vez que se paga una cuota de entrada a desafío se solicitan los datos de tarjeta de crédito o cuenta corriente bancaria. En el caso de cuenta corriente, los datos se almacenan de forma segura y siempre pueden ser modificados.}

\usecase{16}
{Pagando cuota de entrada a desafío. (Participante)}
{Confirmación de datos de pago y confirmación del pago. La comunicación con los sistemas de operaciones bancarias debe ser segura.}

\usecase{17}
{Otorgando créditos a participantes. (Administrador)}
{Les permite ingresar a desafíos sin pagar la cuota. Si ganan, se les resta la cuota del premio.}

\subsubsection{Regionalización}

\usecase{18}
{Consultando un ranking regional. (Participante, Administrador)}
{Rankings de cada nivel. Según el nivel del participante, podrá acceder a uno o más rankings. Siempre puede acceder al de su región. Puede ver estadísticas de todos los participantes y los equipos que fueron usando en los desafíos.}

\usecase{19}
{Accediendo a balances del sitio por región. (Administrador)}
{Información de cuenta y facturación del sitio en cada región, con distintos niveles de granularidad.}

\usecase{20}
{Especificando controles de acceso de usuarios según leyes de la región. (Administrador)}
{Se puede filtrar el acceso total o permitir sólo el acceso a ciertos desafíos: sólo los que son gratuitos y sin premio en dinero, sólo los que son simulados, sólo ciertos deportes, etc. También se puede filtrar acceso a usuarios específicos.}

\subsubsection{Publicidad}

\usecase{21}
{Agregando publicidad en el sitio. (Dueño de Derechos de TV, Representante de Engines Gráficos)}
{ABM para agregar publicidad en los videos de simulación, en el engine 3D/2D, en los videos de partidos reales, o en distintos lugares estratégicos del sitio (a determinar). La interfaz es muy importante por cuestiones de usabilidad de los interesados. El acceso a este ABM está limitado a ciertas IPs y requiere datos de autenticación especiales.}


\subsubsection{Minería de Datos}

\usecase{22}
{Consultando datos estadísticos históricos de usuarios. (Dueño de Derechos de TV, Representante de Engines Gráficos)}
{Se refiere a la consulta de datos estadísticos de los usuarios por parte de inversores y sponsors para ser utilizados en futuras campañas publicitarias, de marketing, etc. Los datos recolectados son de caracter demográfico, valor de las apuestas realizadas, jugador más seleccionado en los equipos, etc.
Todos los datos posibles se pueden consultar y descargar usando datos de autenticación especiales. Se puede filtrar por año, por región, por deporte, etc.}

\usecase{23}
{Consultando datos de preferencia / comportamiento de usuarios. (Dueño de Derechos de TV)}
{Todos los datos posibles se pueden consultar y descargar, usando datos de autenticación especiales. Se pueden obtener usuarios específicos o filtrarlos por deporte, por región, por posición en aĺgún ranking, etc.}
% \newpage
% \section{Análisis de riesgos}
% \subsection{Riesgos de casos de uso}

~

\textbf{Caso de uso \#1:} Simulando desafío de basket según nuevas reglas
\begin{itemize}
\item{\textbf{Riesgo:} La versión anterior del simulador puede no ser lo suficientemente extensible/modificable como para poder incorporar los cambios requeridos, lo cual
puede implicar un profundo rediseño}
\item{\textbf{Contingencia:} Asignar la realización de este caso de uso a la primer etapa de la construcción para evitar posibles retrasos en la fecha de entrega. (No se incluye
en la etapa de la elaboración ya que está relacionado con la funcionalidad y no con la arquitectura)}
\item{\textbf{Probabilidad:} Media}
\item{\textbf{Impacto:} Medio, ya que retrasaría la fecha del release del producto. Los usuarios quieren poder ver estos detalles y nuevas acciones en las simulaciones.}
\end{itemize}

~

\textbf{Caso de uso \#2:} Simulando desafíos de otros deportes
\begin{itemize}
\item{\textbf{Riesgo:} Dominio desconocido: Pueden generarse contratiempos en el desarrollo de los simuladores de otro deporte, al ser escencialmente diferentes al basket}
\item{\textbf{Descripción:} El equipo de desarrollo cuenta con cierta experiencia previa ya que ha implementado el simulador de basket. Sin embargo, el desarrollo de
un simulador para otro deporte puede requerir contemplar situaciones que no han surgido hasta el momento}
\item{\textbf{Mitigación:} Si bien resulta razonable tomar por cota superior el tiempo que ha tomado el desarrollo del simulador de basket para estimar el tiempo de los 
demás simuladores, estimar un adicional del 20\% del tiempo para cualquier eventualidad}
\item{\textbf{Probabilidad:} Media}
\item{\textbf{Impacto:} Medio, ya que retrasaría la fecha del release del producto}
\end{itemize}

~

\textbf{Caso de uso \#7:} Posicionándose en ranking jerárquico. Usuario.
\begin{itemize}
\item{\textbf{Riesgo:} La mala elección de la arquitectura o de la distribución de los servidores puede degradar gravemente la performance. La correcta implementación de este caso de uso es vital para el 'core' de la aplicación, debido a la masiva cantidad de usuarios que tenga el sistema}
\item{\textbf{Contingencia:} Asignarle al desarrollo de este caso de uso un papel central en el proceso. Dedicarle el tiempo necesario (o más) al análisis y la elección 
de una arquitectura adecuada, una posible distribución de servidores y la elección de empresas que nos provean de dichos servidores vs la compra de servidores. Se le ha asignado
un lugar en la 1ra iteración del proceso de elaboración}
\item{\textbf{Probabilidad:} Alta}
\item{\textbf{Impacto:} Alto. Una performance degradada a nivel general puede implicar graves riesgos para el producto}
\end{itemize}

~

\textbf{Caso de uso \#8:} Apostando dinero real en desafíos. Usuario.
\begin{itemize}
\item{\textbf{Riesgo:} Los datos de tarjetas de crédito y bancarios ingresados por los usuarios pueden ser robados o publicados. Se está trabajando con data extremadamente
sensible}
\item{\textbf{Contingencia:} Asignarle al desarrollo de este caso de uso un papel central en el proceso. Dedicarle el tiempo necesario (o más) al análisis y la elección 
de una arquitectura adecuada que permita persistir estos datos utilizando estrictas normas de seguridad. Pedir asesoramiento y trabajar conjuntamente con un equipo de expertos en seguridad informática y encriptación. Se le ha asignado un lugar en la 2da iteración del proceso de elaboración}
\item{\textbf{Probabilidad:} Alta}
\item{\textbf{Impacto:} Alto. El robo o la publicación de datos bancarios puede tener consecuencias muy graves tanto económicas como legales para el proyecto}
\end{itemize}

~

\textbf{Caso de uso \#9:} Controlando acceso de usuarios según leyes de la región respecto a apuestas online
\begin{itemize}
\item{\textbf{Riesgo:} Actuar de forma ilegal en aquellas regiones en donde las apuestas están prohibidas}
\item{\textbf{Contingencia:} Reflexionar en conjunto con el caso de uso \textbf{\#8}} 
\item{\textbf{Probabilidad:} Media}
\item{\textbf{Impacto:} Medio. Pueden generarse inconvenientes económicos y legales}
\end{itemize}

~

\textbf{Caso de uso \#10 y \#11:} Streaming de desafíos
\begin{itemize}
\item{\textbf{Riesgo:} Los usuarios no pueden ver los desafíos debido a que el streaming requiere de una cantidad de datos mayor a la soportada por la conexión}
\item{\textbf{Contingencia:} Asignarle al desarrollo de este caso de uso un papel central en el proceso. Dedicarle el tiempo necesario (o más) al análisis y la elección 
de una arquitectura adecuada que permita transmitir muchos datos. Hacer un análisis profundo del estado de conectividad de cada región en donde desea lanzarse el prodcuto.
Elegir el rendering en 2D/3D en función del análisis. Para el caso del streaming de los partidos reales, elegir la resolución en función del análisis realizado}
\item{\textbf{Probabilidad:} Alta}
\item{\textbf{Impacto:} Medio. Los usuarios desean ver los desafíos en la mayor calidad posible, pero estarán muy disconformes si no pueden hacerlo en absoluto}
\end{itemize}


\subsection{Tabla de análisis de riesgos}

\begin{center}
    \begin{tabular}{ | l | l | l | p{5cm} |}
    \hline
    Impacto | Probabilidad & Alta & Media & Baja \\ \hline
    Alto & (7)-(8) & - & - \\ \hline
    Medio & (10)-(11) & (1)-(2)-(9) & - \\ \hline
    Bajo & - & - & - \\ \hline
    \end{tabular}
\end{center}

% \newpage
% \section{WBS}
% En el proceso de elaboración se incluyen las tareas relacionadas con la arquitectura del sistema, y se genera una base ejecutable del código sobre la cual se pueda 
empezar a testear dicha arquitectura.

Al mismo tiempo se desarrollan los componentes con un rol central dentro de la arquitectura definida, y los módulos de los casos de uso con más riesgos
de alto impacto a nivel estructural, dejando cada una en una iteración diferente. La idea es construir la arquitectura de forma incremental. La opción a nivel arquitectura propuesta
en cada iteración debe ser compatible con la anterior, y al mismo tiempo solucionar el caso de uso actual.

\begin{figure}[h!]
   \includegraphics[scale=0.80]{imagenes/etapas-elaboracion.pdf}
   \caption{División de tareas en la etapa de elaboración}
\end{figure}

En la etapa de construcción se desarrollan los módulos y casos de uso faltantes faltantes de forma iterativa incremental. 

Finalmente en la etapa de transición se realizan tareas de mantenimiento.

\newpage
\begin{landscape}

\begin{figure}[h!]
   \includegraphics[scale=0.80]{imagenes/etapas-construccion.pdf}
   \caption{División de tareas en la etapa de construcción y transición}
\end{figure}

\end{landscape}
\newpage




% \subsection{Estimación de Módulos}
% A continuación se estima en horas hombre los módulos de nivel 1 y 2 del WBS. La estimación se basa en un grupo de trabajo de 4 recursos de 8 horas cada uno, 20 días por mes. Las horas hombre son las horas totales a consumir entre los 4 recursos sumados. Al final se realiza una sumarización de los números para estimar la duración total del proyecto.

\includegraphics[width=\textwidth, page=1, clip, trim=20 0 20 30]{imagenes/estimacionModulos.pdf}

\newpage
\includegraphics[width=\textwidth, page=2, clip, trim=20 200 20 30]{imagenes/estimacionModulos.pdf}

Como puede observarse, el total estimado es bastante razonable para la magnitud del proyecto. Los tiempos podrían acelerarse si se contratara más gente y se tuviera un grupo de trabajo de 2 o 3 personas en cada módulo. Se estima que en 5 meses el proyecto debería estar funcionando, salvando las demoras que puedan causar los proveedores en responder y realizar su trabajo.
% \subsection{Division en etapas e iteraciones}
% \input{division-etapas}
% \section{Primer Entrega}
% No realizamos ninguna correción sobre la primer entrega, por lo que no la volvemos a presentar en esta
% entrega.
% \newpage
% \section{Atributos de calidad}
% En esta sección presentamos los escenarios de calidad que obtuvimos tras analizar los resultados del QAW.

\subsection{Atributos de disponibillidad}

\escenario
{Atributo de disponibilidad}
{El sistema debe estar andando todo el tiempo}
{Externa}
{Solicita acceso al sistema}
{Normal}
{Sistema}
{El sistema responde normalmente}
{Disponibilidad del 99,99\% (se puede caer aprox 1h en todo el año)}


~

\escenario
{Atributo de disponibilidad}
{Si falla un enlace regional, se redirige el tráfico a regiones cercanas de manera uniforme.}
{Servidor regional}
{No responde}
{Normal}
{Sistema}
{El sistema detecta la falla en el servidor regional y redirige el tráfico a las regiones más cercanas. La región entra en modo degradado. Se loguea la falla y se envía una notificación al técnico en redes}
{El servidor / enlace son reparados en menos de 24hs.}


~

\escenario
{Atributo de disponibilidad}
{Se produce una falla en una base de datos de un servidor}
{Interna}
{Falla en la base de datos de un servidor}
{Normal}
{Subsistema de almacenamiento y manejo de datos persistidos}
{El subsistema detecta y loguea la falla. El servidor originial continúa operando normalmente a través del uso de votación. La base de datos cambia a silencioso en el caso de ser la primera falla, y es reemplazada por una nueva instancia en caso de ser la segunda. Se envía una notificación al Data Base Manager.}
{Se garantiza disponibilidad a pesar de una falla en la base de datos el 99.99\% de las veces}

~

\escenario
{Atributo de disponibilidad}
{Se pierden datos de una base de datos de servidor}
{Interna}
{Pérdida de datos en una base de datos de servidor }
{Normal}
{Subsistema de almacenamiento y manejo de datos persistidos}
{El subsistema detecta y loguea la falla. El servidor recupera los datos ya que cuenta con una base redundante. La base de datos cambia a silencioso en el caso de ser la primera falla, y es reemplazada por una nueva instancia en caso de ser la segunda. Se envía una notificación al Data Base Manager.}
{Se mantienen los datos a pesar de una pérdida en una de las bases de datos el 99.99\% de las veces}

~

\escenario
{Atributo de disponibilidad}
{Se pierden los datos de todas las bases de datos de servidor}
{Interna o externa}
{Perdida de datos de todas las bases de datos de servidor}
{Normal}
{Subsistema de almacenamiento y manejo de datos persistidos}
{El subsistema detecta y loguea la falla. El servidor vuelve al último estado consistente de las últimas 4 horas, ya que realiza un backup cada esa cantidad de tiempo. Se restauran todas las bases. Se envía una notificación al Data Base Manager.}
{La probabilidad del escenario anterior es $<$ 0.00001\%. Las bases se restauran en un tiempo $< 4$ horas}

~

\escenario
{Atributo de disponibilidad}
{En cada región habrá varios servidores con una capacidad máxima de usuarios que puede atender, debido a limitaciones de hardware / conexión. Si nuevos usuarios se agregan y superan el 90\% de la capacidad, habrá que agregar un nuevo servidor y balancear la carga}
{Externa}
{Solicita acceso al sistema}
{A 1 pedido de alcanzar el límite de usuarios}
{Servidor regional}
{El servidor responde normalmente. Se incorpora un nuevo servidor a la red, y se aplica el balanceo de carga correspondiente}
{Se realiza la subdivisión de la región agregando un nuevo nodo en menos de 6 horas}

~

\escenario
{Atributo de disponibilidad}
{Enlaces congestionados durante streaming de partido real}
{Externa}
{Disminución de la capacidad de enlace durante streaming de partido real}
{Normal}
{Servidor regional}
{Se detecta el cambio de bitrate. Se realiza un downgrade de la calidad de video. El usuario continúa observando el partido de forma fluída, pero con menor calidad.}
{El streaming del video mantiene un rate constante de cuadros por segundo el 99.99\% de los casos}

~

\escenario
{Atributo de disponibilidad}
{Enlaces congestionados durante streaming de video de simulación}
{Externa}
{Disminución de la capacidad de enlace durante streaming de simulación}
{Normal}
{Servidor regional}
{Se detecta el cambio de ancho de banda del enlace. Se realiza un downgrade en el bitrate}
{El streaming de la simulación mantiene un rate constante de cuadros por segundo el 99.99\% de los casos}

~

\escenario
{Atributo de disponibilidad}
{Se caen enlaces de región durante transmisión de torneo continental o mundial}
{Externa}
{Caída de enlaces de región durante transmisión de torneo continental o mundial}
{Normal}
{Servidor regional}
{Se detecta la caída de los enlaces en la región. Se utiliza la topología de la conexión de regiones para triangular los paquetes y que lleguen a los usuarios.}
{El 99.99\% de las veces el usuario continúa viendo la transmisión del evento sin cortes abruptos, experimentando a lo sumo un cambio en la calidad del video}

\subsection{Atributos de performance}

\escenario
{Atributo de performance}
{Quiere que todo lo respectivo al manejo de dinero (depósitos y retiros de los participantes
via tarjeta de crédito o caja de ahorro) sea super seguro (no quiere papelones y que los
datos de las millones de tarjetas de los participantes aparezcan publicados en Reddit),
transparente y rápido. Que los datos queden resguardados y sólo haya que actualizarlos
esporádicamente.}
{usuario}
{depósito / retiro de dinero}
{operación normal}
{subsistema de pagos}
{el sistema realiza la operación satisfactoriamente}
{el sistema realiza la operación en menos de 15 segundos}

~

\escenario
{Atributo de performance}
{Propone un sistema de bitrate variable automático/manual de los streams de video para
que se pueda bajar la calidad de los videos en base al bandwidth detectado disponible del
usuario.}
{usuario}
{observa transmisión de partido}
{sistema degradado}
{sistema}
{se modifica calidad del video}
{se modifica la calidad del video en menos de 10 segundos}

~

\escenario
{Atributo de performance}
{Quiere que mientras sea posible se use el engine 3d de mayor calidad al 2d.}
{usuario}
{observa simulación de partido}
{sistema degradado}
{dispositivo móvil antiguo}
{se reemplaza engine 3d por 2d}
{antes de comenzar la reproducción de la simulación se reemplaza el engine 3d por 2d}


\subsection{Atributos de seguridad}

\escenario
{Atributo de seguridad}
{Atacante intenta robar datos de tarjetas de crédito o cuentas corrientes bancarias, pero el sistema lo impide}
{Atacante}
{Intenta robar datos de tarjetas de crédito o cuentas corrientes almacenados en servidores}
{Normal}
{Datos del sistema}
{Se detecta y se impide el ataque.}
{Se detecta y se impide el 99\% de los ataques.}

~

\escenario
{Atributo de seguridad}
{Atacante roba datos de tarjetas de crédito o cuentas corrientes bancarias, pero no puede descifrarlos}
{Atacante}
{Roba datos de tarjetas de crédito o cuentas corrientes almacenados en servidores}
{Normal}
{Datos del sistema}
{Se guardan los datos en un formato imposible de leer.}
{Toma más de 1000 años descifrar los datos.}

~

\escenario
{Atributo de seguridad}
{Atacante intercepta comunicación del sistema con el usuario.}
{Atacante}
{Interviene pasivamente una comunicación entre el usuario y el sistema}
{Normal}
{Comunicación del sistema}
{La comunicación está protegida por SSL, con lo cual el contenido de los paquetes es imposible de leer}
{Toma más de 1000 años descifrar los datos interceptados}


~

\escenario
{Atributo de seguridad}
{Atacante se hace pasar por el sistema para robarle datos al usuario.}
{Externa}
{Interviene activamente una comunicación entre el usuario y el sistema, tomando el rol del sistema}
{Normal}
{Comunicación del sistema}
{El sistema utiliza un mecanismo de autenticación del servidor mediante certificados y clave asimétrica. El browser alerta al usuario de que se han vulnerado los certificados SSL. Los paquetes obtenidos por el atacante están encriptados, por lo cual su contenido no puede determinarse}
{Los usuarios advertidos acerca del posible riesgo comienzan una nueva sesión segura el 99.99\% de las veces. En el caso de que el usuario envíe datos sin darse cuenta, toma más de 1000 años descifrar los datos interceptados}

~

\escenario
{Atributo de seguridad}
{Atacante modifica mensajes enviados entre el sistema y el usuario para forzar al sistema a realizar acciones no solicitadas por el usuario.}
{Externa}
{Interviene activamente una comunicación entre el usuario y el sistema, modificando mensajes capturados en el canal de comunicación}
{Normal}
{Comunicación del sistema}
{El sistema utiliza un mecanismo de verificación de integridad de los mensajes recibidos, tanto del lado del cliente como del servidor. El mecanismo de integridad viaja encriptado para evitar que sea modificado.}
{Toma más de 1000 años encontrar un mensaje que estando modificado tenga sentido y verifique la integridad.}

~

\escenario
{Atributo de seguridad}
{Un usuario logueado logra vulnerar el subsistema de pagos y cobros.}
{Usuario identificado}
{Vulnera el subsistema de pagos y cobros y genera movimientos de dinero a su favor}
{Normal}
{Subsistema de pagos y cobros}
{El sistema tiene un audit trail con el registro de todas las acciones realizadas por todos los usuarios logueados y revierte las operaciones realizadas por el usuario.}
{El 99.99\% de las veces el log tiene todos los datos necesarios para revertir las operaciones del usuario.}

~

\escenario
{Atributo de seguridad}
{Usuario no autorizado desea hacer uso de los datos recolectados por minería}
{Usuario sin privilegios de administrador}
{Intento de acceso a datos recolectados por minería}
{Normal}
{Sistema}
{El sistema loguea el intento de acceso. El sistema valida los permisos del usuario en el sistema y posteriormente niega el acceso}
{Los usuarios no autorizados no logran acceder a los datos el 99.99999\% de los casos}

~

\escenario
{Atributo de seguridad}
{Auditor verifica que el código de la simulación y cálculo de resultados de desafíos no se haya modificado}
{Auditor}
{Solicitud de hashes de auditoría para módulos de simulación y cálculo de resultados de desafíos}
{Normal}
{Sistema}
{Se otorgan los hashes correspondientes a ambos módulos}
{La coincidencia de los hashes obtenidos con los conservados con el auditor garantizan que el código no ha cambiado el 99.9999\% de los casos (muy baja probabilidad de colisiones en la función de hash)}

~

\escenario
{Atributo de seguridad}
{Usuario culpa al sistema de que no se le ha asignado el premio de un desafío en el que ha participado y ganado, pero no figura su lista de desafíos}
{Usuario}
{Acusación de premio no otorgado}
{Normal}
{Sistema}
{Se le muestra en base al audit trail del sistema el listado de todas las inscripciones a desafíos que realizó, quedando en evidencia que el usuario no se ha inscripto en dicho desafío}
{El audit trail mantiene una relación 1 a 1 con las operaciones del usuario en un 99.99\% respecto a las acciones del usuario del sistema. Es decir, no hay acciones que no estén logueadas y en el log aparecen únicamente acciones realizadas por dicho usuario}

~

\escenario
{Atributo de seguridad}
{Resolvedor de desafío de liga de fantasía obtiene un dato erróneo de una jugada provisto por el sistema externo que brinda resultados en real-time}
{Externa}
{Dato erróneo del sistema proveedor}
{Normal}
{Sistema}
{La resolución de una jugada en el minuto a minuto de un desafío de liga de fantasía es obtenida a partir de una votación, por lo que el resultado correcto es calculado}
{El proceso de votación obtiene el resultado correcto el 99.99\% de las veces}

\subsection{Atributos de modificabilidad}

\escenario
{Atributo de modificabilidad}
{Quiere ver estadísticas acerca del comportamiento de los participantes de todas las
temporadas (los más ganadores/perdedores en desafíos/dinero, los mejores/peores
equipos formado por participantes de diferentes caps, valores de caps de participantes,
rankings de regiones más ganadoras en desafíos/dinero, el modo de desafío más utilizado
por los participantes, etc...). Lo que está en paréntesis son sólo ejemplos. Quiere que la
mayor cantidad de datos que el sitio maneja pueda ser fácilmente minada por datos. Y
que esos datos pueda estar a cargo de administradores expertos para luego crear
desafíos acordes o otorgar créditos a participantes que califiquen.}
{administrador}
{agregar estadísticas acerca del comportamiento de los participantes}
{tiempo de ejecución}
{sistema}
{se agrega las nuevas estadísticas}
{se agregan las nuevas estadísticas en menos de una hora sin reiniciar el sistema}

~

\escenario
{Atributo de portabilidad}
{Debe de poder correr en la mayor cantidad de plataformas posibles, incluyendo móviles.}
{desarrollador}
{adaptar interfaz a una nueva plataforma}
{tiempo de diseño}
{interfaz de usuario}
{se adapta la interfaz a la nueva plataforma}
{se adaptan los cambios en menos de 50hs}

~

\escenario
{Atributo de modificabilidad}
{Quiere una interfaz similar al representante de empresas con derechos de televisación (Maxi) para controlar publicidades en las simulaciones y el sitio en general.}
{administrador}
{modificar publicidad}
{tiempo de ejecución}
{interfaz para el manejo de publicidades}
{se hace modificación de las publicidades}
{se muestran las nuevas publicidades en menos de 20 segundos}

~

\escenario
{Atributo de modificabilidad}
{Plantea necesidad de regionalizar la plataforma, debido a lo limitado y la mala calidad del
hardware/servidores disponibles para la plataforma en las regiones iniciales, sobre todo
teniendo en cuenta que los streams de video pasan por “dentro” del sistema. No se está
hablando de tercerizar el servicio a sitios como Vimeo o Youtube, sino que el tráfico pase
de alguna manera por los servidores del sitio. Los enlaces físicos/hardware subyacente
implica una cantidad de usuarios limitada / máxima por servidor}
{Interna}
{Incorporación una nueva región al sistema}
{normal}
{sistema}
{se incorpora una nueva región}
{el setup de la configuración se hace en menos de 5 horas}

~

\escenario
{Atributo de modificabilidad}
{También se quiere aumentar el caudal de redes sociales que se utilizan para “afectar” las estadísticas (no solo Twitter, sino incorporar Facebook,Google+, etc.)}
{desarrollador}
{incorporar una nueva red social al sistema}
{tiempo de diseño}
{sistema}
{se incorpora la nueva red social}
{se emplean menos de 40 hs}

~

\escenario
{Atributo de modificabilidad}
{Quiere que el énfasis se de en mejorar el módulo de simulación, para que sea lo más real
posible. Tiene contacto con asociaciones de jugadores e incluso jugadores, técnicos y
periodistas deportivos de diferentes deportes, para ayudar a mejorar el motor de
reglas/simulación. También tiene contacto con empresas de redes sociales/sentiment
analysis, para ayudar a interfacear con las mismas, y mejorar el módulo de
menciones/popularidad de los jugadores. El espíritu es que toda la simulación pueda irse
mejorando poco a poco hasta que represente lo mejor posible la realidad sin que sea un
dolor de cabeza introducir cambios.}
{sponsor, stakeholder}
{agregar nuevas reglas/acciones al motor de simulación}
{tiempo de diseño}
{motor de simulación}
{reglas/acciones nuevas agregadas sin efectos secundarios}
{se invierten menos de 10hs hombre}


\subsection{Atributos de usabilidad}

\escenario
{Atributo de usabilidad}
{Quiere poder que él y sus administradores de confianza puedan ver un dashboard en
tiempo real del estado de cuenta del sitio de cada una de las regiones y niveles (incluye locales, continentales, global, etc...) y de cualquier grupo de participantes.}
{administrador}
{ver estado de cuenta del sitio}
{tiempo de ejecución}
{sistema}
{el sistema provee un dashboard con el estado de cuenta del sitio de cada una de las regiones}
{el administrador es capaz de administrar / comprender la información suministrada en menos de 5 minutos}

~

\escenario
{Atributo de usabilidad}
{Aunque no es su responsabilidad, le interesa que la interfaz gráfica de usuarios tenga la
calidad de un “juego”, sobre toda al momento de ver a los jugadores, las jugadas de los
técnicos, colocar el nombre y logo del equipo del participante, etc... con animaciones y
efectos especiales con aceleración gráfica (blurs, iluminación dinámica, depth of field, etc).}
{usuario}
{el usuario desea administrar su equipo}
{operación normal}
{Interfaz web / Móvil}
{se muestra una interfaz gráfica llena de animaciones en donde el usuario puede administrar su equipo }
{el test de usabilidad supera el 95\% de satisfacción}

~

\escenario
{Atributo de usabilidad}
{Usabilidad del subsistema controlador de publicidades en las simulaciones y en el sitio}
{Usuario administrador de publicidades}
{Desea minimizar el impacto de sus errores al configurar publicidades}
{Runtime}
{Sistema}
{Se provee un botón de cancelación para volver a la configuración anterior}
{Los cambios realizados se vuelven atrás y las publicidades no cambian.}

~

\escenario
{Atributo de usabilidad}
{Usabilidad del subsistema controlador de publicidades en las simulaciones y en el sitio}
{Usuario administrador de publicidades}
{Desea estar seguro de dónde se muestra en el sistema la publicidad que está modificando}
{Runtime}
{Sistema}
{Se provee una previsualización de cada publicidad, ubicada en el entorno gráfico que corresponde, acompañada de un texto descriptivo.}
{Durante 1 hora se le enseña a dos personas a usar la interfaz de ABM de publicidades y luego se les solicita hacer cambios en todas las publicidades. Modificarán correctamente al menos el 80\%.}

~

\escenario
{Atributo de usabilidad}
{Usabilidad del subsistema controlador de publicidades en las simulaciones y en el sitio}
{Usuario administrador de publicidades}
{Desea insertar la misma publicidad en muchos lugares del sistema de forma eficiente}
{Runtime}
{Sistema}
{Se provee una funcionalidad especial para cargar una publicidad eligiendo múltiples sitios.}
{Toma a lo sumo 3 clicks adicionales cargar una publicidad en muchos sitios que cargarla en un único sitio (además de todos los clicks necesarios para seleccionar los distintos sitios) (suponiendo que no se cometen errores en la selección de sitios).}

~

\escenario
{Atributo de usabilidad}
{Usabilidad del subsistema de pagos y cobros}
{Usuario}
{Desea estar tranquilo de que ingresar en el sistema su número de tarjeta de crédito o cuenta corriente es seguro.}
{Runtime}
{Sistema}
{Se muestran los nombres de las autoridades que auditaron la seguridad del sistema y los documentos que lo prueban.}
{En menos de 5 minutos el usuario se anima a ingresar sus datos.	}


% \escenario
% {Atributo de auditabilidad}
% {}
% {administrador}
% {buscar operaciones hechas con tarjeta de crédito}
% {operación normal}
% {sistema}
% {se muestra la operación buscada}
% {cada vez que se realiza una operación con tarjeta de crédito se guarda quién la hizo, en qué momento, en qué ip y qué monto se acreditó / retiro}














































% \section{Arquitectura}
% En esta sección presentamos la arquitectura que proponemos para cumplir con los atributos de calidad
ya planteados.

\subsection{Diagrama de nivel 0}
\begin{figure}[H]
  \centering
  \includegraphics[width=\textwidth]{imagenes/Nivel0.png}
  \caption{Diagrama de Arquitectura de nivel 0.}
\end{figure}
\newpage

\subsection{Diagrama de nivel 1}
\begin{figure}[H]
  \centering
  \includegraphics[width=\textwidth]{imagenes/Nivel1.png}
  \caption{Diagrama de Arquitectura de nivel 1.}
\end{figure}

El sistema se compone de N de estas instancias. Cada una representa el software que corre en un servidor. N representa la cantidad de regiones (mínima unidad geográfica) en donde corre el sistema. Cada instancia está deployada en un servidor independiente, ubicado en dicha región. Cada uno de estos servidores tiene una réplica que corre en modo shadow por si el servidor primario falla.
El sistema se compone también del conjunto de componentes que corre en los dispositivos de los usuarios.

Cada servidor almacena datos locales a su región. Las estadísticas que almacena, la información de desafíos, la información de sus usuarios son en su gran mayoría regionales, con la excepción
de que dicho servidor esté siendo utilizando para algún desafío como nodo intermedio en el árbol de jerarquía de desafíos extra-regionales. (Ver documento jeraraquía global). En ese caso,
almacena y propaga los datos del desafío que está simulando.

La comunicación entre el cliente y el servidor se realiza de 4 formas diferentes: Streaming de partido real, obtención de datos para rendering de simulación, realizar una compra de fichas
para apostar o extracción de dinero y finalmente pedidos de consulta web comunes como ser: registración, login, consulta de ranking, creación de equipo, participación de desafío, etc. Llamamos
'pedido general' a esta última categoría. Cada una de estas 4 formas de comunicación utiliza un conector distinto ya que necesitan satisfacer diferentes requerimientos.
Estos conectores tienen un estilo call-return para poner énfasis en que la comunicación es sincrónica: Un usuario hace un pedido de streaming. A partir de ese momento se inicia una
sesión entre el servidor y el usuario para mandar un flujo de datos a través del canal de comunicación que dura hasta que finaliza la conexión.

El subsistema de desafíos maneja tanto las simulaciones como las ligas de fantasía.

'Currification' tiene un acuerdo con las empresas televisivas proveedoras de transmisiones, mediante la cual permite crear enlaces para satisfacer los pedidos de los usuarios para ver
partidos televisados. El manejador de pedidos recibe los datos de cada transimisión a través de un pipe, y resuelve a través de un mapping sesiones de usuarios vs transmisiones para
saber qué transmisiones debe enviarle a cada usuario. Una situación análoga sucede con las simulaciones.

Los stakeholders pueden agregar publicidades a través de una interfaz gráfica intuitiva y simple. Esta les permite decidir en qué lugar mostrarán cada publicidad: UI de la web,
rendering de simulaciones o streaming de partidos. Por este motivo, el subsistema de publicidad rutea las nuevas publicidades a los subsistemas correspondientes.

Los controles de auditoría se hacen sobre los subsistemas de pagos y cobros (para corroborar que no haya movimientos irregulares de dinero), y sobre el subsistema de simulación de
desafíos, verificando que el hash del código de simulación no haya sido modificado.

La autenticación y restricción al sitio por parte de los usuarios se realiza a partir de reglas: Algunas direcciones de IP están restringidas. Correspondientes a aquellas zonas en donde
están prohibidas las apuestas web. Por otro lado, cuando un usuario asocia una tarjeta a su cuenta el sistema verifica que la tarjeta sea local a la región a la cual se encuentra.
Finalmente, cuando un usuario quiere realizar una compra utilizando una tarjeta, esta última debe coincidir con aquella que ha sido registrada previamente.


\newpage

\subsection{Jerarquia Virtual}
\begin{figure}[H]
  \includegraphics[width=\textwidth, page=1, clip, trim=20 0 20 110]{imagenes/jerarquia-global.pdf}
\end{figure}
\newpage
\begin{figure}[H]
  \includegraphics[width=\textwidth, page=2, clip, trim=20 20 20 10]{imagenes/jerarquia-global.pdf}
\end{figure}

\begin{figure}[H]
  \centering
  \includegraphics[width=\textwidth]{imagenes/nivel1-simulaciones.png}
\end{figure}

\newpage

\subsection{Subsistema de Desafios}
\begin{figure}[H]
  \includegraphics[width=1.1\textwidth, page=1, clip, trim=10 0 10 0]{imagenes/Subs-desafios.pdf}
  \caption{Subsistema de Desafios.}
\end{figure}
% \newpage
El subsistema de desafíos se encarga de crear desafíos, inscribir participantes, iniciar desafios en el momento indicado, proveer detalles y estado de cada desafio a los usuarios.
Cuando un desaífo comienza se encarga de informar al subsistema de simulación o liga de fantasia seúgn corresponda. Además se encarga de cobrar créditos de inscripción y de pagar premios (tanto en créditos como premios especiales, que luego los usuarios podrán cambiar por el premio real en dinero o lo que corresponda. También provee una interfaz de consulta de estadísticas de desafios.\\

\begin{itemize}
\item Desafíos Terminados: Los desafíos finalizados durante la última semana van a tener más solicitudes de consulta de estado. Es por esto que esos desafíos se mantienen
en el repositorio principal de desafíos con redundancia activa para mejorar la disponibilidad y la performance. Pasados los 7 días de finalizado, un proceso que ejecuta una vez por día se encarga de depurar el repositorio principal para mejorar la performance de las búsquedas, asumiendo que esos desafíos serán consultados con menos frecuencia.

\item Creador de Desafíos: Devuelve todas las opciones posibles para construir un desafío (tanto para modo simulado como para modo liga de fantasía). Luego recibe la configuración elegida y crea el desafío. Se almacena en un repositorio local que luego se propaga a todas las réplicas regionales.

\item Inscripción de participantes: Muestra una lista con todos los desafíos por cada deporte (tanto simulados como liga de fantsía) que aún estén con la inscripcion abierta (aún no comenzaron ajugarse). Por cada uno muestra el valor en creditos para ingresar. Luego se inscribe un participante (siempre y cuando ls créditos le alcancen). Para mejorar la performance y disponibilidad (muchos podíran intentar inscribirse a la vez), luego de hacer el checkeo se guardan en una caché todos los pedidos de  inscripcion que luego se van persistiendo por procesos dedicados.

\item Renderizador de estado y detalles: Devuelve el estado de un desafio especifico (si está abierto a inscripciones, si se esta jugando o si está terminado) asi como también la cuenta regresiva para que empiece el desafio (y se cierren inscripciones) y todos los detalles asociados (premios, cuota de inscripción, tipo de desafío, formato de torneo o fechas que se juegan, etc.), y el ranking, si corresponde. Internamente se utiliza una caché. Cada pedido que llega se guarda en un repositorio y se solicitan los datos a un manager de cache (que se crea en el momento que se necesita) y a un proceso que busca los datos en los repositorios persistentes (tambéin se crea cuando se necesita). Si el dato llega de la caché, se mata al proceso que  fue a disco (y tambien se mata al proceso de la caché). En cambio si no estaba en la caché, el dato llegaár del disco (y ahí tambéin se guarda en la caché para soportar eficentemente un período corto de muchos pedidos de detales del mismo desafio. Un proceso se ejecuta cada 1 minuto y borra los datos de caché que tengan más de un minuto de antigüedad. Dado que los datos no deben mostrarse en tiempo real, este mcanismo mejora la disponibildad y la performance de respuesta a muchas solicitudes de detalles de un desafío popular.
\end{itemize}

\begin{figure}[H]
  \includegraphics[width=\textwidth, page=3, clip, trim=20 0 20 30]{imagenes/Subs-desafios.pdf}
  \caption{Zoom en Procesador Desafio Terminado y Creador de Desafios}
\end{figure}

\begin{figure}[H]
  \includegraphics[width=\textwidth, page=4, clip, trim=20 0 20 0]{imagenes/Subs-desafios.pdf}
  \caption{Zoom en Inscriptor de Participantes y Renderizador de Estados y Detalles Desafio Activo}
\end{figure}

\newpage

\subsection{Subsistema de Simulacion}
\begin{figure}[H]
  \includegraphics[width=1.1\textwidth, page=1, clip, trim=10 0 10 0]{imagenes/subs-simulacion.pdf}
  \caption{Subsistema de Simulacion.}
\end{figure}

\begin{figure}[H]
  \includegraphics[width=1.1\textwidth, page=2, clip, trim=25 0 10 0]{imagenes/subs-simulacion.pdf}
  \caption{Enviador Datos a servidores externos.}
\end{figure}

\begin{figure}[H]
  \includegraphics[width=1.1\textwidth, page=3, clip, trim=25 250 10 0]{imagenes/subs-simulacion.pdf}
  \caption{Receptor de datos de Simulacion.}
\end{figure}

\newpage

\subsection{Subsistema de Ligas de Fantasia}
\begin{figure}[H]
  \centering
  \includegraphics[width=\textwidth]{imagenes/fantasia.png}
  \caption{Subsistema de Ligas de Fantasia.}
\end{figure}

El procesador de comienzo de desafío almacena el desafío nuevo en el repositorio de desafíos activos, junto con todos los partidos a tener en cuenta.
Luego crea un procesador de comienzo de partido, que se encarga de revisar si el desafío indicado tiene algún partido que comenzar, y en caso que no sea así, programa un timer para
que le avise cuándo comienza el próximo partido. En este esquema, se crea un procesador de comienzo de partido para cada desafío nuevo que llega.
Cada vez que hay un partido, el procesador de comienzo de partido crea un procesador de minuto a minuto de partido. Este último, cada un minuto,
solicita al Subsistema de Estadisticas de Partidos las últimas actualizaciones. Con ellas evalúa las reglas de puntajes y actualiza los rankings (por ejemplo, una regla podría ser "Si mete gol, suma 3 puntos", y si una actualización
indica que Agüero metió un gol, entonces se envia al ranking A todos los participantes cuyo equipo tenga a Agüero suma 3 puntos).
Una vez actualizado los rankings, vuelve aprogramar el timer por un minuto. De esta manera habrá un procesador minuto a minuto por cada partido activo, que en cada minuto
actualizará los puntajes del ranking.\\

El procesador de comienzo de partido, luego de crear el de minuto a minuto, se programa
para el próximo partido. Si no hay más partidos, simplemente se destruye a sí mismo.\\

Cuando cada partido termina, el procesador de minuto a minuto se destruye a sí mismo, pero antes avisa del fin de partido al procesador de fin de partido, que verifica si
fue el último del desafio y en tal caso informa de la finalización del partido al subsistema de desafíos.

\newpage

\subsection{Subsistema de Ranking}
\begin{figure}[H]
  \centering
  \includegraphics[width=\textwidth]{imagenes/Subsistema-ranking.png}
  \caption{Subsistema de Rankings.}
\end{figure}

\newpage

\subsection{App Cliente}
\begin{figure}[H]
  \centering
  \includegraphics[width=\textwidth]{imagenes/Cliente.png}
  \caption{Applicacion del lado del Cliente.}
\end{figure}

\newpage

\subsection{Manejador pedidos Cliente}
\begin{figure}[H]
  \centering
  \includegraphics[width=\textwidth]{imagenes/Manejador-de-pedidos.png}
  \caption{Manejador de pedidos realizados por el cliente.}
\end{figure}

\newpage

\subsection{Subsistema de Usuarios}
\begin{figure}[H]
  \centering
  \includegraphics[width=\textwidth]{imagenes/manejo-usuarios.png}
  \caption{Subsistema de manejo de usuarios.}
\end{figure}

Cuando un usuario se registra, el manejador
ingresa la información en la base de datos.
Para interactuar con el sistema, el usuario
inicia una sesión.
Ante cada pedido, el sistema autoriza que el
usuario que generó el pedido tenga suficientes
permisos para hacerlo

\newpage

\subsection{Subsistema de Cobro y Pagos}
\begin{figure}[H]
  \centering
  \includegraphics[width=\textwidth]{imagenes/subs-cobro-y-pago.png}
  \caption{Subsistema de cobros y pagos.}
\end{figure}

Los datos de tarjetas y cuentas y el Tokenizador corren en una maquina distinta a todo el resto. Esto se hace para aislarla lo maximo posible y evitar cualquier vulnerabilidad que pueda llegar a tener el resto del sistema. Ademas estas maquinas deberan tener seguridad fisica, para evitar posibles robos y/o violaciones fisicas al sistema.

Funcionamiento:
\begin{enumerate}
\item {
  \begin{itemize}
  \item Caso medio de pago nuevo:
  Al manejador le llegan pedidos de pagos y/o cobros, autenticados y con informacion de un medio de pago nuevo. Entonces le pasamos esta informacion a el Tokenizador, que le genera un token, persiste en base de datos la informacion y el token correspondiente y le devuelve el token al Manejador de pedidos.
  \item Caso medio de pago existente:
  Al manejador le llegan pedidos de pagos o cobros, autenticados y con un id o refencia minima (elegida en por el Cliente) de que medio de pago se usara. Entonces con ese id o referencia, le pedimos al Tokenizador y obtenemos el token correspondiente.
  \end{itemize}
}

\item Luego con el token correspondiente, se lo brindamos al Realizador de pedidos, que entiende de tokens y con el se comunica con el medio de pago correspondiente y realiza la accion pertinente brindandole el token al medio de pago.

\item Luego con el resultado de la accion, la logeamos en el registro de movimiento y le devolvemos el resultado de la accion al Realizador y este al Manejador
\end{enumerate}

Ademas la información mas sensible (Numero de Tarjeta o Cuenta completa, codigo de seguridad de la tarjeta, etc) seran almacenada encriptada, mediante un algoritmo de clave asimetrica, donde la clave para desencriptar la tengan unicamente los dueños del sistema.

De esta forma, cumpliriamos con un estandar de seguridad llamado PCI, el cual es necesario para para poder realizar cobros y pagos con tarjetas de credito y cuentas bancarias. Este estandar sera todo el tiempo testeado para verificar que estemos siempre cumpliendo y en norma, ya que en caso de no estarlo estariamos abierto a posibles ataques y/o robos de datos.

\subsubsection{Biografia consultada}
\begin{itemize}
\item \href{https://www.pcisecuritystandards.org/documents/Tokenization_Guidelines_Info_Supplement.pdf}{Norma PCI.}\\
\item \href{https://www.quora.com/Do-companies-like-Amazon-etc-have-a-server-farm-to-store-creditcard-information-on-database}{Como cuidan sus datos compañias del estilo Amazon.}
\end{itemize}

\newpage

\subsection{Subsistema de Streaming}
\begin{figure}[H]
  \centering
  \includegraphics[width=\textwidth]{imagenes/subs-streaming.png}
  \caption{Subsistema de streaming de partidos reales.}
\end{figure}

La interacción entre el manejador de pedidos y el componente de streaming de partidos reales es análoga al de las simulaciones.
Los usuarios envían un pedido al manejador en donde solicitan ver el partido 'i'.
El manajador guarda una entrada en el repositorio en donde agrega una entrada que establece que el usuario 'u' está subscripto
al streaming del partido 'i'.

El manager de pedidos de streaming crea entonces una entrada en el 'repositorio de streams solicitados'. Este repositorio es 
consultado por el receptor de desafíos para saber a qué streamings generados por los proveedores de transmisiones suscribirse. 

Hay un conector custom por cada proveedor de servicios diferente

\newpage

\subsection{Subsistema de Publicidad}
\begin{figure}[H]
  \centering
  \includegraphics[width=\textwidth]{imagenes/Subsistema-de-publicidad.png}
  \caption{Subsistema de Publicidades.}
\end{figure}

La idea es que el manejador de pedidos stakeholders le indique al persisitidor de publicidades qué publicidad desea
guardar, indicándole en dónde deber mostrarse (transmisión de un partido, simulación o página principal del sitio) y en qué momento (hora).
Luego, el obtenedor de publicidades consulta periodicamente (cada 3 segundos) el repositorio solicitándole publicidades que deban mostrarse
en el momento en que consulta (+/- 3 segundos) y las envía a traves de un router a cada componente que hará uso del mismo (subsistema de
 simulación, subsistema de streaming y manejador de pedidos de usuario)

\newpage

\subsection{Subsistema de Estadisticas de Partidos}
\begin{figure}[H]
  \centering
  \includegraphics[width=\textwidth]{imagenes/Subsistema-de-estadistica-de-partido.png}
  \caption{Subsistema de Estadisticas de Partidos.}
\end{figure}

Los proveedores estadisticos que tenemos contratados publican actualizaciones periodicas de los partidos que se están disputando.
Como los datos publicados son propensos a errores decidimos utilizar un voter central que recibe las salidas de los múltiples procesadores y decide el
resultado correcto en función de los votos. Estos resultados son persisitidos en un repositorio el cual puede ser accedido mediante el obtenedor de
estadísticas de partidos.

\newpage

% \section{Arquitectura TP1}
% A continuación presentamos lo que seria un especie de Diagrama de Nivel 1 sobre la Arquitectura de la solución
del TP1.

\begin{figure}[H]
  \centering
  \includegraphics[width=\textwidth]{imagenes/TP1-Arquitectura.png}
  \caption{Arquitectura del TP1.}
\end{figure}

La idea basica es muy similar a la arquitectura propuesta en este trabajo practico, tenemos varios clientes
que nos realizan pedidos, en este caso serian consultas a los Rankings, consultas a las estadisticas de jugadores
y tecnicos o acciones respecto al sistema de desafios (creacion de cuentas, creacion de equipos, creacion de desafios,
manejo de fichas, etc). Todos estos pedidos son recibidos por el Manejador de Pedidos, el cual redirige cada pedido
al componente correspondiente.\\
Por otro lado en el componente Subsistema de Usuarios y Desafios, como ya dije estara toda la logica sobre el tema de
la creacion de equipos, desafios, etc. Este componente luego le brindara la informacion necesaria del desafio(Equipos, Jugadores, Tecnicos) al Subsistema de Simulacion para que este pueda simular el desafio. Luego una vez resuelta la simulacion, el subsistema de simulacion devuelve el log del partido para que el Subsitema de Usuarios y Desafios haga lo correspondiente, por un lado devolverle el resultado y el log al Manejador de pedidos para que le llegue al cliente, y por otro lado registrar el resultado, actualizar rankings, actualizar cap y pagar apuestas (en caso de ser necesario), etc.\\
El Subsistema de Simulacion para poder simular el desafio consume por un lado las estadisticas historicas de los jugadores y por otro lado las menciones en Twitter, para asi de esta forma tener los valores correspondientes a la hora de calcular los umbrales de exito.\\

Como podemos apreciar hay conceptos compartidos entre las dos Arquitecturas propuestas, pero la del Trabajo Practico 1 es mucho mas simple y chica, ya que la aplicacion en si era mas chica y menos compleja. No tenemos nada sobre ligas de fantasia, dinero real, tarjetas de credito, streaming de partidos reales, engines 2d y 3d para ver la simulacion, etc. Basicamente no teniamos ningun atributo de calidad por cumplir a raja tabla, por lo que es un Arquitectura pensada mucho mas en la funcionalidad que en otras cosas.

% \section{Comparaciones y Conclusiones}
% \subsection{UP vs Scrum}
Desde el punto de vista de la planificación, UP y Scrum se parecen en que ambas hacen un desgloce en tareas de cada uno de los módulos/user stories a realizar en la siguiente etapa/sprint. Sin embargo, UP realiza asignación de horas y tareas a cada recurso de antemano, mientras que en Scrum cada desarrollador elige la user storie que va a realizar cuando desea, siempre y cuando se llegue con los tiempos de entrega (finalización del sprint). En ambas metodologías se priorizan actividades: módulos o casos de uso en UP mediante el análisis de riesgos y user stories en Scrum mediante la estimación de esfuerzo y valor de negocio.

Desde el punto de vista de la ejecución de las actividades, UP las organiza en etapas mientras que Scrum en sprints. Estos difieren principalmente en que UP podría tener distinta duración de las iteraciones de distintas etapas (pero las iteraciones de una misma etapa duran lo mismo), mientras que en Scrum los sprints son todos de la misma duración. Además en UP se planifican los módulos que se tratarán en cada etapa desde el principio, mientras que en Scrum se elige el sprint backlog justo antes de cada sprint (de hecho podrían agregarse stories nuevas al backlog).

Scrum garantiza entregas en poco tiempo (idealmente al final de cada sprint), mientras que al final de las iteraciones de UP no necesariamente se pueden realizar entregas. Por otro lado, UP permite realizar una planificación a largo plazo de gran parte del proyecto, mientras que Scrum se va adaptando sobre la marcha y se actualiza el futuro del proyecto antes de cada sprint.


\subsection{Programming in the Small vs Programming in the Large}
Programming in the Small consiste en el modelado de sistemas pequeños, donde todos los detalles son conocidos de antemano y se puede modelar con un gran nivel de detalle. Los requerimientos no funcionales son poco exigentes: pocos usuarios, no hay grandes problemas de performance o disponibilidad; la modificabilidad se logra principalmente desde el diseño orientado a objetos; seguridad y usabilidad son los únicos atributos de calidad que podrían considerarse más en detalle.

Por otro lado, Programming in the Large consiste en el modelado de sistemas muy grandes, que no pueden diseñarse por completo usando clases y objetos, sino que se debe hacer un diseño de alto nivel, partir el problema en módulos (WBS) y atacar los módulos por separado y con una planificación detallada. Los atributos de calidad juegan un rol fundamental ya que la performance y disponibilidad del sistema suelen ser atributos muy importantes y difíciles de satisfacer. Además el dinero juega un papel importante ya que en la mayoría de las veces marcará la diferencia en la mejoría de la calidad del sistema.

PitL requiere una planificación mucho más detallada que PitS, ya que al tratarse de un proyecto de mayor duración teporal y de mayor consumo de recursos es importante que el uso de estos recursos se haga de manera eficiente. Los cambios de requerimientos en un proyecto grande van a generar un mayor impacto dependiendo de en qué fase del proyecto se realicen, es por esto que es fundamental la elicitación exhaustiva en las primeras etapas para evitar cambios de último momento que atrasen la finalización. En cambio las modificaciones en requerimientos de un proyecto pequeño se pueden tratar en todo momento dado que un impacto grande siempre estará limitado por el tamaño del proyecto que sigue siendo pequeño.

\subsection{Conclusiones}
En general para proyectos de PitS conviene utilizar metodologías ágiles como Scrum, ya que permiten generar entregables rápidamente. En un proyecto PitL las entregas a corto plazo pueden no ser factibles al principio porque quizás haya que construir muchos módulos antes de poder ver algún resultado interesante. Además Scrum permite el cambio de requerimientos antes de cada sprint, mientras que en UP se espera que los cambios de requerimientos sean sólo al principio del proyecto. Pero esto último no sirve para PitS dado que los cambios en requerimientos podrían surgir en cualquier momento, luego de que el cliente vea un entregable y cambie de opinión con respecto a cosas que había pedido previamente.

Por otro lado, los proyectos de PitL se acoplan mejor con UP ya que este provee un marco de planificación detallada muy necesario para proyectos grandes, que permite llevar el control global del proyecto. El esfuerzo y los recursos consumidos en la planificación son muy necesarios para PitL para evitar problemas futuros: el análisis y mitigación de riesgos es la clave para evitar problemas. En cambio Scrum prioriza actividades según valor de negocio, que en PitL podrían ser muchas y luego de varios meses de desarrollo podrían surgir problemas en módulos que no eran prioritarios para el negocio pero sí riesgosos para el funcionamiento correcto (por ejemplo, cuestiones de seguridad). Además en PitL no se esperan grandes cambios en los requerimientos (al menos no a nivel global, siempre podría haber pequeños cambios en los módulos), con lo cual Scrum sería demasiado flexible en este aspecto.

Por ende en general veremos Scrum aplicado a proyectos pequeños con requerimientos cambiantes y priorización según valor de negocio, y UP a proyectos grandes que requieren planificación a largo plazo y priorización según riesgos para evitar problemas.



% Ejemplo insertar imagen desde un pdf (mas facil cuando hay muchas paginas)

% \begin{figure}[H]
%   \includegraphics[width=1.1\textwidth, page=3, clip, trim=25 250 10 0]{imagenes/subs-simulacion.pdf}
%   \caption{Receptor de datos de Simulacion.}
% \end{figure}

% Ejemplo de insertar imagen png

% \begin{figure}[H]
%   \centering
%   \includegraphics[width=\textwidth]{imagenes/Subsistema-de-estadistica-de-partido.png}
%   \caption{Subsistema de Estadisticas de Partidos.}
% \end{figure}


\section{Reentrega}
\subsection{Administrador de Usuarios.}

\begin{figure}[H]
   \centering
   \includegraphics[width=\textwidth]{reentrega/imagenes/usuarios.png}
   \caption{Administrador de Usuarios.}
\end{figure}

En este componente, se reciben y manejan muchos tipos de pedidos, todos enfocados en el usuario. Los pedidos nos pueden llegar de dos fuentes distintas, del \texttt{Administrador de desafios}, que nos manda pedidos de ver los equipos de un usuario, de crear un equipo nuevo, de ver la cantidad de fichas, de descontarle fichas o de sumarle fichas a un usuario. El resto de los pedidos, nos llegan del \texttt{Receptor de pedidos} que nos manda pedidos de login, registro, agregar medio de pago, eliminar medio de pago, mostrar medios de pagos ya registrados, comprar fichas, vender o retirar fichas, ver cantidad de fichas, ver equipos, crear equipos, entre otros.

Todos estos pedidos son recibidos por el \texttt{Manejador de pedidos}, que los comprende y actuando como un switch, fowardea el pedido al componente correspondiente.

Por otro lado tenemos un componente \texttt{Encriptador}, el cual se encarga de dada cierta información que le pasan, encriptarla o desencriptarla mediante un algoritmo asimetrico, utilizando la clave publica y la clave privada del Sistema. Asi de esta forma podemos encriptar información sensible, para luego persistirla ya encriptada y que unicamente pueda ser desencriptada con la clave privada del Sistema. Ademas contamos con el componete \texttt{Transmisor de datos}, que representa un simple pasamanos con el \texttt{Administrador de Datos}.

Entonces como dijimos, a este componente le pueden llegar una amplia variedad de pedidos, a continuacion detallaremos como es el flujo de atencion de cada uno de estos pedidos:

En el caso de un pedido de login, se le fowardea el pedido con el usuario y contraseña al \texttt{Autenticador de usuarios}. Este le pide al \texttt{Encriptador}, que nos encripte la contraseña y luego le pide al \texttt{Transmisor de datos} la contraseña que tenemos persistida en el sistema (la cual ya esta encriptada), entonces comparamos ambas contraseñas encriptadas y si estas son iguales, el usuario se autentico con exito, por lo que se le devuelve un codigo de autenticacion exitosa, en caso de ser distintas, se devuelve un codigo de autenticacion rechaza. Esta respuesta es para el \texttt{Manejador de pedidos}, que se encargara de propagar la respuesta a quien corresponda.

En el caso de un pedido de registro, le fowardeamos el pedido al \texttt{Registrador de usuario}, que le pide al \texttt{Encriptador} que le encripte la contraseña que nos llega y luego se encarga de persistir en el \texttt{Transmisor de Datos}, el nuevo usuario. En caso de que surga algun error en el proceso se devolvera un codigo de Registro Rechazado y sino un codigo de Registro Ok.

A al hora de recibir un pedido de agregar o registrar un nuevo medio de pago, se le fowardea el pedido al \texttt{Administrador de medios de pago}, que lo primero que realiza es comprobar que el medio de pago que se desea agregar se corresponde con el pais el cual esta registrado el usuario, para lo cual le consulta esta información al \texttt{Administrador de informacion basica del usuario}. En caso de que no se corresponda con el pais donde se registro el usuario, se rechaza la operacion. Luego una vez validado esto, se comunica con el Proveedor correspondiente y le pasamos la información que nos brindo el usuario. De esta forma el proveedor nos validara que es correcta la información que tenemos y ademas nos provee un token de identificacion para este medio de pago. En caso de fallar esta validación por parte del \texttt{Proveedor de medios de pagos}, se le rechaza la operacion. Una vez obtenido este token, lo tomamos junto a cierta información minima que nos permite reconocer el medio de pago (por ejemplo los ultimos 4 numeros de la tarjeta y el nombre del proveedor) y encriptamos esta informacion con ayuda del \texttt{Encriptador}. Luego una vez encriptada esta información, le pedimos al \texttt{Transmisor de datos} que persista este nuevo medio de pago. Una vez finilazado esto con exito, se devuelve Ok.

De la misma forma, un usuario puede querer eliminar un medio de pago que registro con anterioridad. En este caso, el usuario primero consultara ver los medios de pagos que ya tiene registrado. Esto dispara un pedido de mostrar los medios de pagos, que le llega al \texttt{Administrador de medios de pagos}, que a su vez se lo enviara al \texttt{Transmisor de Datos}. Este le obtendra el listado de la información basica de cada medio de pago que tiene asociado este usuario y se le propagara el listado hasta llegarle el usuario. Luego el usuario seleccionara que desea eleiminar uno de esos medios de pago . Esto se refleja en otro pedido que nos llega, de eliminar un medio de pago, con la información pertienente al medio de pago solicitado. Entonces el \texttt{Administrador de medios de pago} unicamente debe fowardear el pedido al \texttt{Transmisor de datos} de eliminar el medio de pago seleccionado.

Una vez que el cliente tiene registrado un medio de pago, este puede comprar y retirar/vender fichas. En el caso de que desee comprar fichas, nos llega un pedido con una cantidad de fichas que desea comprar. Este pedido le llega al \texttt{Administrador de fichas}, el cual le pide al \texttt{Administrador de medios de pago} el listado de todos los medios de pagos que tiene asociado el cliente, asi le da a elegir al cliente cual medio de pago desea utilizar. Una vez el cliente selecciono uno. Desde el \texttt{Administrador de fichas} le pedimos al \texttt{Administrador de medios de pagos} que le cobre al medio de pago correspondiente la plata asociada a la cantidad de fichas que el usuario pidio. Este último interactura con el \texttt{Proveedor de medios de pagos} correspondiente, utilizando el token que tenemos persistido en nuestro sistema. Si el \texttt{Proveedor de medio de pagos} nos rechaza la operación, entonces le rechazamos la operación al usuario. En cambio, si la operacion es aprobada el \texttt{Administrador de fichas}, le pide al \texttt{Transmisor de datos} que le agregue la nueva cantidad de fichas disponibles al usuario y se le responde que la operación fue exitosa.

Se da un caso similar, si el cliente quiere retirar o vender sus fichas. Primero nos llega un pedido de retirar una determinada cantidad de fichas al \texttt{Administrador de fichas}, igual que el caso anterior se le da a elegir al usuario que medio de pago quiere utilizar para recibir el dinero. Una vez que este lo eligio, le pedimos al \texttt{Administrador de medios de pagos}, que le deposite la plata correspondiente en ese medio de pago comunicandose con el \texttt{Proveedor de Medios de Pagos}. Si el proveedor nos rechaza la operación, entonces le rechazamos la operacion al usuario, en el caso contrario le pedimos al \texttt{Transmisor de datos} que le descuente la cantidad de fichas correspondientes al usuario.

Surgen dos casos muy parecidos a los de comprar y vender fichas, que son los pedidos que nos llegan del \texttt{Administrador de desafios}, que serian descontar fichas y agregar fichas. Estos simplificaciones de los anteriores, ya que unicamente el \texttt{Administrador de fichas} habla con el \texttt{Transmisor de datos} para restar o agregar fichas al usuario, ya que al tratarse del cobro de inscripción en desafios o el pago de premios de desafios, no necesitamos interactuar con ningun medio de pago.

Por otro lado llegan los pedidos de creación de equipos, que puede llegar tanto por el \texttt{Receptor de pedidos} o desde el \texttt{Administrador de desafios}. Este pedido entonces es fowardeado al \texttt{Administrador de equipos} que primero obtiene la información de los jugadores disponibles (en el caso de venir de un desafio tendra un scope de jugadores que permite el desafio) atravez del \texttt{Transmisor de datos}, para asi mostrarle al usuario los jugadores que puede elegir. Una vez que este selecciona los jugadores que quiere que integren su equipo. Nos llega un nuevo pedido, con estos por parametro y el \texttt{Administrador de equipos} le pide al \texttt{Transmisor de datos} que persista este nuevo equipo relacionado al usuario.

Ademas tenemos todos los pedidos que son únicamente de visualizar recursos, por ejemplo, ver cantidad de fichas, ver equipos ya creados, ver medios de pagos ya registrados o ver información basica del usuario (tipo perfil). Todos estos se fowardean a su componete correspondiente, es decir, si el pedido es de información basica al \texttt{Administrador de información basica}, si es de fichas al \texttt{Administrador de fichas}, etc. Entonces este componete se encargara de pasarle el pedido al \texttt{Transmisor de datos} para que nos obtenga los recursos solicitados y luego se le devuelven al usuario.



\newpage

\bibliographystyle{plain}
\bibliography{tp3}

\end{document}

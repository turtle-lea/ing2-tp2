\documentclass[a4paper]{article}
\usepackage[spanish]{babel}
\usepackage[utf8]{inputenc}
\usepackage{graphicx}
\usepackage{pdfpages}
\usepackage{enumerate}
\usepackage{listings}
\usepackage{color}
\usepackage{indentfirst}
\usepackage{fancyhdr}
\usepackage{latexsym}
\usepackage[colorlinks=true, linkcolor=black]{hyperref}
\usepackage{wrapfig}
\usepackage{algpseudocode}
\usepackage{calc}
\usepackage{amsmath, amsthm, amssymb}
\usepackage{amsfonts}
\usepackage{lscape}
\definecolor{gray}{gray}{0.5}
\definecolor{light-gray}{gray}{0.95}
\definecolor{orange}{rgb}{1,0.5,0}

\input{page.layout}
\usepackage{underscore}
\usepackage{caratula}
\usepackage{url}

\newcommand{\cod}[1]{{\tt #1}}
\newcommand{\negro}[1]{{\bf #1}}
\newcommand{\ital}[1]{{\em #1}}
\newcommand{\may}[1]{{\sc #1}}
\newcommand{\tab}{\hspace*{2em}}

\newcommand{\sprintstory}[6]{\begin{tabular}{| p{3cm} | p{12cm} |}
 \hline
 TargetProcess ID: & #1 \\
 \hline
 User Story: & #2 \\
 \hline
 Esfuerzo estimado: & #3 \\
 \hline
 Business Value: & #4 \\
 \hline
 Descripción: & #5 \\
 \hline
 Criterios de\newline Aceptación: & #6 \\
 \hline
\end{tabular}}

\newcommand{\usecase}[3]{\noindent\textbf{CU\##1. #2}\\
#3\\
~\\
}

\newenvironment{taskstable}
{ \begin{tabular}{| p{14cm} | p{1cm} |}
 \hline
 \multicolumn{2}{|c|}{{\bf División en tareas}}\\
 \hline
 {\bf Tarea} & {\bf HH}\\
 \hline }
{ \end{tabular} }

\newcommand{\task}[2]{#1 & #2\\
 \hline}

\hypersetup{
 pdfstartview= {FitH \hypercalcbp{\paperheight-\topmargin-1in-\headheight}},
 pdfauthor={Grupo},
 pdfsubject={Dise\~{n}o}
}

\lstset{escapechar=@}

\begin{document}

\thispagestyle{empty}
\materia{Ingeniería de Software II}
\submateria{Primer Cuatrimestre de 2016}
\titulo{Trabajo Práctico II: The Curry Game release v7.1.2}

\integrante{Leandro Matayoshi}{79/11}{leandro.matayoshi@gmail.com}
\integrante{Matías Pizzagalli}{257/12}{matipizza@gmail.com}
\integrante{Gastón Requeni}{400/11}{grequeni@hotmail.com}
\integrante{Martín Santos}{413/11}{martin.n.santos@gmail.com}

\makeatletter

\maketitle

\newenvironment{myindentpar}[1]
{\begin{list}{1}
         {\setlength{\leftmargin}{#1}}
         \item[]
}
{\end{list}}

\newpage
\section{Casos de uso}
A continuación presentamos el diagrama de casos de uso simplificado (sólo mostramos relaciones entre casos de uso y actores, sin especificar dependencias).

\begin{figure}[h!]
  \centering
  \includegraphics[width=\textwidth]{imagenes/casosDeUso.png}
  \caption{Diagrama de Casos de Uso}
\end{figure}

\newpage

\subsection{Descripción de los casos de uso}

\subsubsection{Cuenta de usuario}
\usecase{01}
{Registrándose en la aplicación. (Participante)}
{Registro de participantes en la aplicación donde les solicitamos sus datos. Esto es necesario para poder autenticarse luego y poder usar la aplicación.}

\usecase{02}
{Autenticándose en la aplicación. (Participante, Administrador)}
{Único medio de ingreso a la aplicación para participar en desafíos o para realizar tareas administrativas (en el caso del Administrador).}

\usecase{03}
{Accediendo a estado de cuenta de participante. (Participante, Administrador)}
{Consulta/modificación de datos personales, dinero ganado, dinero perdido, estadísticas de desafíos ganados, equipos armados, etc. Un Administrador puede acceder con los mismos privilegios que el Participante a cuentas de participantes (como help desk o para extraer datos útiles).}

\subsubsection{Desafíos modo Simulación}

\usecase{04}
{Creando un desafío modo simulación. (Participante, Administrador)}
{Elección del tipo de torneo (plaoffs, liga, combinado zonas, etc), duración, deporte, cantidad máxima de participantes, fecha de inicio. Según la región del participante y según tenga acceso a niveles superiores, podrá elegir una región o se tomará la región del participante para mostrar el desafío a otros participantes de la misma región. Una vez creado empieza la cuenta regresiva para su comienzo. Usa el armado de equipo (CU\#10). Se indica premio si corresponde.}

\usecase{05}
{Participando en desafío modo simulación. (Participante)}
{Elección de un desafío de un listado de desafíos de la región. Usa el pago de la cuota de entrada (CU\#16). Usa también el armado de equipo (CU\#10). Al finalizar el desafío, ganará premios si corresponde.}

\usecase{06}
{Mirando partido simulado en tiempo real. (Participante)}
{Motor de simulación según deporte. Generación de un stream que se traduce a video o se reproduce usando el motor 3D o 2D según se determine en tiempo de ejecución. En cuanto al simulador de basquet, habrá que extenderlo según nuevos requerimientos. El resto de los motores se crean desde cero.}

\subsubsection{Desafíos modo Liga de Fantasía}

\usecase{07}
{Creando desafío modo liga de fantasía. (Participante, Administrador)}
{Elección de la cantidad de fechas reales que abarca, deporte, cantidad máxima de participantes, fecha de inicio. Según la región del participante y según tenga acceso a niveles superiores, podrá elegir una región o se tomará la región del participante para mostrar el desafío a otros participantes de la misma región. Una vez creado empieza la cuenta regresiva para su comienzo. Usa el armado de equipo (CU\#10).}

\usecase{08}
{Participando en desafío modo liga de fantasía. (Participante)}
{Elección de un desafío de un listado de desafíos de la región. Usa el pago de la cuota de entrada (CU\#16). Usa también el armado de equipo (CU\#10). Al finalizar el desafío, ganará premios si corresponde. Los resultados de cada fecha se calculan en base a las estadísticas reales.}

\usecase{09}
{Mirando en vivo partidos de ligas reales. (Participante)}
{Transmisión provista por dueños de los derechos de televisación en vivo de partidos reales.}


\subsubsection{Desafíos (simulados o liga de fantasía)}

\usecase{10}
{Armando equipo. (Participante)}
{Elección de jugadores en cada posición de la cancha y de los suplentes (difiere según deporte del desafío). Elección de un técnico para tener una estrategia de jugada o para sumar puntos según estadísticas (en el caso de liga de fantasía).}

\usecase{11}
{Consultado estado actual de un desafío. (Participante)}
{Estado del partido actual tomado de la simulación o tomado de las estadísticas en tiempo real del partido real. Estado del torneo en general, usando resultados de las simulaciones o los resultados de fechas anteriores de la liga de fantasía. Incluye ranking de participantes.}

\subsubsection{Chat}

\usecase{12}
{Chateando con participantes del desafío. (Participante)}
{Servicio de mensajería que permita intercambiar mensajes entre participantes de un mismo desafío.}

\usecase{13}
{Enviando IM a otro participante. (Participante)}
{Servicio de mensajería instantánea entre dos participantes.}

\subsubsection{Pagos y Cobros}

\usecase{14}
{Ingresando datos de tarjeta de crédito. (Participante)}
{La primera vez que se paga una cuota de entrada a desafío se solicitan los datos de tarjeta de crédito o cuenta corriente bancaria. En el caso de tarjeta de crédito, los datos se almacenan de forma segura y siempre pueden ser modificados.}

\usecase{15}
{Ingresando datos de cuenta corriente. (Participante)}
{La primera vez que se paga una cuota de entrada a desafío se solicitan los datos de tarjeta de crédito o cuenta corriente bancaria. En el caso de cuenta corriente, los datos se almacenan de forma segura y siempre pueden ser modificados.}

\usecase{16}
{Pagando cuota de entrada a desafío. (Participante)}
{Confirmación de datos de pago y confirmación del pago. La comunicación con los sistemas de operaciones bancarias debe ser segura.}

\usecase{17}
{Otorgando créditos a participantes. (Administrador)}
{Les permite ingresar a desafíos sin pagar la cuota. Si ganan, se les resta la cuota del premio.}

\subsubsection{Regionalización}

\usecase{18}
{Consultando un ranking regional. (Participante, Administrador)}
{Rankings de cada nivel. Según el nivel del participante, podrá acceder a uno o más rankings. Siempre puede acceder al de su región. Puede ver estadísticas de todos los participantes y los equipos que fueron usando en los desafíos.}

\usecase{19}
{Accediendo a balances del sitio por región. (Administrador)}
{Información de cuenta y facturación del sitio en cada región, con distintos niveles de granularidad.}

\usecase{20}
{Especificando controles de acceso de usuarios según leyes de la región. (Administrador)}
{Se puede filtrar el acceso total o permitir sólo el acceso a ciertos desafíos: sólo los que son gratuitos y sin premio en dinero, sólo los que son simulados, sólo ciertos deportes, etc. También se puede filtrar acceso a usuarios específicos.}

\subsubsection{Publicidad}

\usecase{21}
{Agregando publicidad en el sitio. (Dueño de Derechos de TV, Representante de Engines Gráficos)}
{ABM para agregar publicidad en los videos de simulación, en el engine 3D/2D, en los videos de partidos reales, o en distintos lugares estratégicos del sitio (a determinar). La interfaz es muy importante por cuestiones de usabilidad de los interesados. El acceso a este ABM está limitado a ciertas IPs y requiere datos de autenticación especiales.}


\subsubsection{Minería de Datos}

\usecase{22}
{Consultando datos estadísticos históricos de usuarios. (Dueño de Derechos de TV, Representante de Engines Gráficos)}
{Se refiere a la consulta de datos estadísticos de los usuarios por parte de inversores y sponsors para ser utilizados en futuras campañas publicitarias, de marketing, etc. Los datos recolectados son de caracter demográfico, valor de las apuestas realizadas, jugador más seleccionado en los equipos, etc.
Todos los datos posibles se pueden consultar y descargar usando datos de autenticación especiales. Se puede filtrar por año, por región, por deporte, etc.}

\usecase{23}
{Consultando datos de preferencia / comportamiento de usuarios. (Dueño de Derechos de TV)}
{Todos los datos posibles se pueden consultar y descargar, usando datos de autenticación especiales. Se pueden obtener usuarios específicos o filtrarlos por deporte, por región, por posición en aĺgún ranking, etc.}
\newpage
\section{Análisis de riesgos}
\subsection{Riesgos de casos de uso}

~

\textbf{Caso de uso \#1:} Simulando desafío de basket según nuevas reglas
\begin{itemize}
\item{\textbf{Riesgo:} La versión anterior del simulador puede no ser lo suficientemente extensible/modificable como para poder incorporar los cambios requeridos, lo cual
puede implicar un profundo rediseño}
\item{\textbf{Contingencia:} Asignar la realización de este caso de uso a la primer etapa de la construcción para evitar posibles retrasos en la fecha de entrega. (No se incluye
en la etapa de la elaboración ya que está relacionado con la funcionalidad y no con la arquitectura)}
\item{\textbf{Probabilidad:} Media}
\item{\textbf{Impacto:} Medio, ya que retrasaría la fecha del release del producto. Los usuarios quieren poder ver estos detalles y nuevas acciones en las simulaciones.}
\end{itemize}

~

\textbf{Caso de uso \#2:} Simulando desafíos de otros deportes
\begin{itemize}
\item{\textbf{Riesgo:} Dominio desconocido: Pueden generarse contratiempos en el desarrollo de los simuladores de otro deporte, al ser escencialmente diferentes al basket}
\item{\textbf{Descripción:} El equipo de desarrollo cuenta con cierta experiencia previa ya que ha implementado el simulador de basket. Sin embargo, el desarrollo de
un simulador para otro deporte puede requerir contemplar situaciones que no han surgido hasta el momento}
\item{\textbf{Mitigación:} Si bien resulta razonable tomar por cota superior el tiempo que ha tomado el desarrollo del simulador de basket para estimar el tiempo de los 
demás simuladores, estimar un adicional del 20\% del tiempo para cualquier eventualidad}
\item{\textbf{Probabilidad:} Media}
\item{\textbf{Impacto:} Medio, ya que retrasaría la fecha del release del producto}
\end{itemize}

~

\textbf{Caso de uso \#7:} Posicionándose en ranking jerárquico. Usuario.
\begin{itemize}
\item{\textbf{Riesgo:} La mala elección de la arquitectura o de la distribución de los servidores puede degradar gravemente la performance. La correcta implementación de este caso de uso es vital para el 'core' de la aplicación, debido a la masiva cantidad de usuarios que tenga el sistema}
\item{\textbf{Contingencia:} Asignarle al desarrollo de este caso de uso un papel central en el proceso. Dedicarle el tiempo necesario (o más) al análisis y la elección 
de una arquitectura adecuada, una posible distribución de servidores y la elección de empresas que nos provean de dichos servidores vs la compra de servidores. Se le ha asignado
un lugar en la 1ra iteración del proceso de elaboración}
\item{\textbf{Probabilidad:} Alta}
\item{\textbf{Impacto:} Alto. Una performance degradada a nivel general puede implicar graves riesgos para el producto}
\end{itemize}

~

\textbf{Caso de uso \#8:} Apostando dinero real en desafíos. Usuario.
\begin{itemize}
\item{\textbf{Riesgo:} Los datos de tarjetas de crédito y bancarios ingresados por los usuarios pueden ser robados o publicados. Se está trabajando con data extremadamente
sensible}
\item{\textbf{Contingencia:} Asignarle al desarrollo de este caso de uso un papel central en el proceso. Dedicarle el tiempo necesario (o más) al análisis y la elección 
de una arquitectura adecuada que permita persistir estos datos utilizando estrictas normas de seguridad. Pedir asesoramiento y trabajar conjuntamente con un equipo de expertos en seguridad informática y encriptación. Se le ha asignado un lugar en la 2da iteración del proceso de elaboración}
\item{\textbf{Probabilidad:} Alta}
\item{\textbf{Impacto:} Alto. El robo o la publicación de datos bancarios puede tener consecuencias muy graves tanto económicas como legales para el proyecto}
\end{itemize}

~

\textbf{Caso de uso \#9:} Controlando acceso de usuarios según leyes de la región respecto a apuestas online
\begin{itemize}
\item{\textbf{Riesgo:} Actuar de forma ilegal en aquellas regiones en donde las apuestas están prohibidas}
\item{\textbf{Contingencia:} Reflexionar en conjunto con el caso de uso \textbf{\#8}} 
\item{\textbf{Probabilidad:} Media}
\item{\textbf{Impacto:} Medio. Pueden generarse inconvenientes económicos y legales}
\end{itemize}

~

\textbf{Caso de uso \#10 y \#11:} Streaming de desafíos
\begin{itemize}
\item{\textbf{Riesgo:} Los usuarios no pueden ver los desafíos debido a que el streaming requiere de una cantidad de datos mayor a la soportada por la conexión}
\item{\textbf{Contingencia:} Asignarle al desarrollo de este caso de uso un papel central en el proceso. Dedicarle el tiempo necesario (o más) al análisis y la elección 
de una arquitectura adecuada que permita transmitir muchos datos. Hacer un análisis profundo del estado de conectividad de cada región en donde desea lanzarse el prodcuto.
Elegir el rendering en 2D/3D en función del análisis. Para el caso del streaming de los partidos reales, elegir la resolución en función del análisis realizado}
\item{\textbf{Probabilidad:} Alta}
\item{\textbf{Impacto:} Medio. Los usuarios desean ver los desafíos en la mayor calidad posible, pero estarán muy disconformes si no pueden hacerlo en absoluto}
\end{itemize}


\subsection{Tabla de análisis de riesgos}

\begin{center}
    \begin{tabular}{ | l | l | l | p{5cm} |}
    \hline
    Impacto | Probabilidad & Alta & Media & Baja \\ \hline
    Alto & (7)-(8) & - & - \\ \hline
    Medio & (10)-(11) & (1)-(2)-(9) & - \\ \hline
    Bajo & - & - & - \\ \hline
    \end{tabular}
\end{center}

\newpage
\section{WBS}
En el proceso de elaboración se incluyen las tareas relacionadas con la arquitectura del sistema, y se genera una base ejecutable del código sobre la cual se pueda 
empezar a testear dicha arquitectura.

Al mismo tiempo se desarrollan los componentes con un rol central dentro de la arquitectura definida, y los módulos de los casos de uso con más riesgos
de alto impacto a nivel estructural, dejando cada una en una iteración diferente. La idea es construir la arquitectura de forma incremental. La opción a nivel arquitectura propuesta
en cada iteración debe ser compatible con la anterior, y al mismo tiempo solucionar el caso de uso actual.

\begin{figure}[h!]
   \includegraphics[scale=0.80]{imagenes/etapas-elaboracion.pdf}
   \caption{División de tareas en la etapa de elaboración}
\end{figure}

En la etapa de construcción se desarrollan los módulos y casos de uso faltantes faltantes de forma iterativa incremental. 

Finalmente en la etapa de transición se realizan tareas de mantenimiento.

\newpage
\begin{landscape}

\begin{figure}[h!]
   \includegraphics[scale=0.80]{imagenes/etapas-construccion.pdf}
   \caption{División de tareas en la etapa de construcción y transición}
\end{figure}

\end{landscape}
\newpage




\subsection{Estimación de Módulos}
A continuación se estima en horas hombre los módulos de nivel 1 y 2 del WBS. La estimación se basa en un grupo de trabajo de 4 recursos de 8 horas cada uno, 20 días por mes. Las horas hombre son las horas totales a consumir entre los 4 recursos sumados. Al final se realiza una sumarización de los números para estimar la duración total del proyecto.

\includegraphics[width=\textwidth, page=1, clip, trim=20 0 20 30]{imagenes/estimacionModulos.pdf}

\newpage
\includegraphics[width=\textwidth, page=2, clip, trim=20 200 20 30]{imagenes/estimacionModulos.pdf}

Como puede observarse, el total estimado es bastante razonable para la magnitud del proyecto. Los tiempos podrían acelerarse si se contratara más gente y se tuviera un grupo de trabajo de 2 o 3 personas en cada módulo. Se estima que en 5 meses el proyecto debería estar funcionando, salvando las demoras que puedan causar los proveedores en responder y realizar su trabajo.
\newpage
\section{Plan de proyecto}
\begin{enumerate}
  \item \Large{\textbf{Elaboración}}
  \begin{enumerate}
    \item \large{\textbf{Iteración \#0}}
    \begin{enumerate}
      \item \normalsize{\textbf{Reconocimiento de casos de uso}}
      \item \normalsize{\textbf{Priorización de casos de uso}}
      \item \normalsize{\textbf{Estimación de tiempos de casos de uso}}
      \item \normalsize{\textbf{Realización de WBS}}
      \item \normalsize{\textbf{Análisis de riesgos}}
    \end{enumerate}
    \item \large{\textbf{Iteración 2}}
    \item \large{\textbf{Iteración 3}}
  \end{enumerate}
  \item \Large{\textbf{Construcción}}
  \item \Large{\textbf{Transición}}
\end{enumerate}

Al momento de la entrega de este documento la iteración \#0 ya ha sido completada. En consecuencia, se incluye en este informe las conclusiones y resultados obtenidos en esta
primera iteración.


\newpage

\bibliographystyle{plain}
\bibliography{tp3}

\end{document}

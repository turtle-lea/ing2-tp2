\subsection{UP vs Scrum}
Desde el punto de vista de la planificación, UP y Scrum se parecen en que ambas hacen un desgloce en tareas de cada uno de los módulos/user stories a realizar en la siguiente etapa/sprint. Sin embargo, UP realiza asignación de horas y tareas a cada recurso de antemano, mientras que en Scrum cada desarrollador elige la user storie que va a realizar cuando desea, siempre y cuando se llegue con los tiempos de entrega (finalización del sprint). En ambas metodologías se priorizan actividades: módulos o casos de uso en UP mediante el análisis de riesgos y user stories en Scrum mediante la estimación de esfuerzo y valor de negocio.

Desde el punto de vista de la ejecución de las actividades, UP las organiza en etapas mientras que Scrum en sprints. Estos difieren principalmente en que UP podría tener distinta duración de las iteraciones de distintas etapas (pero las iteraciones de una misma etapa duran lo mismo), mientras que en Scrum los sprints son todos de la misma duración. Además en UP se planifican los módulos que se tratarán en cada etapa desde el principio, mientras que en Scrum se elige el sprint backlog justo antes de cada sprint (de hecho podrían agregarse stories nuevas al backlog).

Scrum garantiza entregas en poco tiempo (idealmente al final de cada sprint), mientras que al final de las iteraciones de UP no necesariamente se pueden realizar entregas. Por otro lado, UP permite realizar una planificación a largo plazo de gran parte del proyecto, mientras que Scrum se va adaptando sobre la marcha y se actualiza el futuro del proyecto antes de cada sprint.


\subsection{Programming in the Small vs Programming in the Large}
Programming in the Small consiste en el modelado de sistemas pequeños, donde todos los detalles son conocidos de antemano y se puede modelar con un gran nivel de detalle. Los requerimientos no funcionales son poco exigentes: pocos usuarios, no hay grandes problemas de performance o disponibilidad; la modificabilidad se logra principalmente desde el diseño orientado a objetos; seguridad y usabilidad son los únicos atributos de calidad que podrían considerarse más en detalle.

Por otro lado, Programming in the Large consiste en el modelado de sistemas muy grandes, que no pueden diseñarse por completo usando clases y objetos, sino que se debe hacer un diseño de alto nivel, partir el problema en módulos (WBS) y atacar los módulos por separado y con una planificación detallada. Los atributos de calidad juegan un rol fundamental ya que la performance y disponibilidad del sistema suelen ser atributos muy importantes y difíciles de satisfacer. Además el dinero juega un papel importante ya que en la mayoría de las veces marcará la diferencia en la mejoría de la calidad del sistema.

PitL requiere una planificación mucho más detallada que PitS, ya que al tratarse de un proyecto de mayor duración teporal y de mayor consumo de recursos es importante que el uso de estos recursos se haga de manera eficiente. Los cambios de requerimientos en un proyecto grande van a generar un mayor impacto dependiendo de en qué fase del proyecto se realicen, es por esto que es fundamental la elicitación exhaustiva en las primeras etapas para evitar cambios de último momento que atrasen la finalización. En cambio las modificaciones en requerimientos de un proyecto pequeño se pueden tratar en todo momento dado que un impacto grande siempre estará limitado por el tamaño del proyecto que sigue siendo pequeño.

\subsection{Conclusiones}
En general para proyectos de PitS conviene utilizar metodologías ágiles como Scrum, ya que permiten generar entregables rápidamente. En un proyecto PitL las entregas a corto plazo pueden no ser factibles al principio porque quizás haya que construir muchos módulos antes de poder ver algún resultado interesante. Además Scrum permite el cambio de requerimientos antes de cada sprint, mientras que en UP se espera que los cambios de requerimientos sean sólo al principio del proyecto. Pero esto último no sirve para PitS dado que los cambios en requerimientos podrían surgir en cualquier momento, luego de que el cliente vea un entregable y cambie de opinión con respecto a cosas que había pedido previamente.

Por otro lado, los proyectos de PitL se acoplan mejor con UP ya que este provee un marco de planificación detallada muy necesario para proyectos grandes, que permite llevar el control global del proyecto. El esfuerzo y los recursos consumidos en la planificación son muy necesarios para PitL para evitar problemas futuros: el análisis y mitigación de riesgos es la clave para evitar problemas. En cambio Scrum prioriza actividades según valor de negocio, que en PitL podrían ser muchas y luego de varios meses de desarrollo podrían surgir problemas en módulos que no eran prioritarios para el negocio pero sí riesgosos para el funcionamiento correcto (por ejemplo, cuestiones de seguridad). Además en PitL no se esperan grandes cambios en los requerimientos (al menos no a nivel global, siempre podría haber pequeños cambios en los módulos), con lo cual Scrum sería demasiado flexible en este aspecto.

Por ende en general veremos Scrum aplicado a proyectos pequeños con requerimientos cambiantes y priorización según valor de negocio, y UP a proyectos grandes que requieren planificación a largo plazo y priorización según riesgos para evitar problemas.

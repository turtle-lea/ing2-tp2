\subsection{Procesador de Solicitudes HTTP iniciales, geolocalización de IPs y monitoreo de servidores de nivel 5} \label{seccionDNS}

% gráfico DNS

El cliente ingresa la URL del sitio. Esto dispara una consulta DNS del navegador
que retorna aleatoriamente la IP de alguno de los servidores de nivel 1.

Luego el cliente se conecta al servidor de nivel 1 haciendo un GET de HTTP. Esto
lo recibe el $"$Receptor de solicitud global$"$, que lee la IP fuente y se la pasa al
$"$Seleccionador de servidor de nivel 5 según IP$"$. Este primero obtiene el GeoPath
usando el $"$Obtenedor de GeoPath según IP$"$ y luego busca en el repositorio de servidores
de nivel 5 todos los que estén atendiendo clientes en esa ubicación geográfica y elige uno
de ellos al azar. Luego retorna la DNS de ese servidor de nivel 5. El $"$Receptor de solicitud
global$"$ luego envía un mensaje de $"$Redirect$"$ de HTTP con la nueva DNS. Esto en el cliente
desencadena una nueva consulta DNS por la IP del servidor de nivel 5 correspondiente, y el
flujo de comunicación continúa con el cliente enviando un GET de HTTP al servidor de nivel 5.

El $"$Obtenedor de GeoPath según IP$"$ busca en el repositorio de GeoPaths por IPs. Este
repositorio está en cada uno de los servidores de nivel 1 y contiene una tabla con todos los
rangos de IP del mundo mapeados a su ubicación geográfica usando GeoPath. Este repositorio
se actualiza todos los días a las 04:00hs de la zona horaria donde el servidor se encuentra
físicamente. Para actualizarse, se consulta un servicio externo de geolocalización que indica los
cambios que hubo a nivel mundial. En el caso extremo que una IP es consultada pero aún no
está en este repositorio, el $"$Obtenedor de GeoPath según IP$"$ detectará la ausencia de la misma
y entonces ingresará el pedido en el repositorio de $"$IPs nuevas sin GeoPath$"$. Este repositorio
funciona como blackboard e inmediatamente despierta el proceso de actualización de IPs,
que consulta únicamente la IP solicitada y la escribe en el repositorio de IPs según GeoPath, para
que el otro proceso la consuma (se queda haciendo polling hasta que aparece). Se usa un
repositorio porque podría haber más de una IP nueva al mismo tiempo, y en tal caso cuando el proceso
de actualización se despierta, actualiza todas las que haya en simultáneo.

Para que este esquema funcione, debemos registrar un nombre para cada servidor de cada nivel en
el sistema DNS. Por ejemplo, si nuestro dominio fuera www.currygame.com, deberíamos registrar todas
las IPs de servidores de nivel 1 a ese nombre. Asumismos que DNS distribuye uniformemente
las consultas entre las distintas IPs (por ejemplo, siempre elige una random con distribución uniforme).
Luego cada servidor específico debe tener un nombre específico, por ejemplo:

\begin{itemize}
	\item ar.n1.s2.currygame.com - Nivel 1 de Argentina, server 2.
	\item ar.n5.s40.currygame.com - Nivel 5 de Argentina, server 40 (que podría ser uno de CABA, BsAs).
\end{itemize}

Esto permite que el cliente luego pueda ingresar directamente usando la URL específica de su región
y no necesite hacer la redirección.
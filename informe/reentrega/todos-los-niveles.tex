\subsection{Esquema general de interacción entre niveles y administrador de pedidos de nivel i}

\begin{figure}[H]
   \centering
   \makebox[\textwidth][c]{\includegraphics[width=1.2\textwidth]{reentrega/imagenes/todos-los-niveles-macro.png}}
   \caption{Vista de todos los niveles en simultaneo.}
\end{figure}

\begin{figure}[H]
   \centering
   \makebox[\textwidth][c]{\includegraphics[width=1.2\textwidth]{reentrega/imagenes/todos-los-niveles-administrador-pedidos.png}}
   \caption{Zoom Administrador de pedido.}
\end{figure}

Un administrador de pedidos es el encargado de responder consultas provenientes de niveles inferiores respecto a consultas de desafíos y ranking general. Cuando el administrador de datos tiene la información suficiente para responder al pedido, el
mismo se resuelve en el momento. Por ejemplo, supongamos que Juan es un usuario de Córdoba y crea un desafío con id = 35.
Si Pedro también es cordobés, al preguntar por el desafío de id=35 el administrador de pedidos de nivel 4 (provincia=Córdoba) va a contar con la información necesaria para responder.
Distinto es el caso en el que Pedro consulta por un desafío que fue creado por un administrador a nivel nacional. En este caso, los datos van a ser manejados por el administrador de pedidos de nivel 3 (país=Argentina). En este caso, se forwardea el
pedido para que sea resuelto por este último.
El mismo mecanismo se produce de manera recursiva hasta llegar al nivel global.

Los administradores pueden crear en desafíos en cada nivel, cuyos datos serán almacenados en el administrador de pedidos
correspondientes. De esta manera se guardan datos de desafíos de los mejores a nivel mundial en el adm. nivel 1 y a nivel continental en el nivel 2. A nivel país, los administradores pueden crear desafíos que excedan a las provincias. Como por ejemplo, un gran DT en donde a priori se sabe que va a participar todo el país. Si esto fuera almacenado a nivel 4, todo el país debería acceder a ese mismo nodo de nivel 4.

El componente ``Procesador de Solicitudes HTTP iniciales, geolocalización de IPs y monitoreo de servidores de nivel 5'' sólo se utiliza a nivel 1 para redirigir usuarios que acceden por primera vez por DNS al servidor de nivel 5 correspondiente a su zona.

Una diferencia entre el nivel 4 y los demás es que existe comunicación horizontal entre los administradores del mismo nivel. Por ejemplo, si Juan creo un desfío en Córdoba y Rodrigo de Tucumán quiere ver el listado de desafíos Cordobeses para poder inscribirse en alguno, los administradores de pedidos cuentan con un bus de comunicación sin necesidad de utilizar
al padre (de nivel 3: país=Argentina) como intermediario. Esto se hace para no sobrecargar de pedidos al componente
que responde pedidos a nivel país. Por el contrario, como ya se mencionó anteriormente, si Rodrigo quiere ver el resultado de la fecha de Gran DT, el componente administrador de nivel 4 deberá forwardear el pedido al de nivel 3.

Otra diferencia entre el nivel 4 y los demás es que en el componente administrador de usuarios dejan de tener sentido algunos componentes como por ejemplo el autenticador o el registrador de usuario ya que los usuarios nunca se van a registrar en los niveles superiores.

A nivel país y continental no hay comunicación a nivel horizontal entre cualquier par de nodos, sólo puede haber comunicación entre hermanos del mismo padre. Por ejemplo, no se puede comunicar el nodo de Argentina con el de España, pero sí el de Argentina con el de Uruguay ya que ambos pertenecen al nodo padre ``Sudamérica''. Los pedidos de regiones superiores siempre se propagan hacia arriba. Rodrigo, de Tucumán puede consultar: desafíos nacionales de Argentina, desafíos sudamericanos y desafíos mundiales. El pedido es forwardeado hasta el nivel que corresponde y luego desciende hasta el
manejador de nivel 5 que generó el pedido. Para evitar sobrecargar a los niveles superiores (que potencialmente son los que pueden recibir mayor cantidad de pedidos), las respuestas se van cacheando en los niveles intermedios. Por ejemplo,
si todos los usuarios de Sudamérica están preguntado por los resultados del desafío mundial de Básket los resultados
se cachean en los administradores de nivel 4 (provincias), de nivel 3(país) y de nivel 2 (continente=Sudamérica), durante un tiempo máximo configurable.


El administrador de pedidos de nivel i también es el encargado de propagar los datos de los renderizadores para simulación, el streaming de partidos reales y los datos de las ligas reales para que los desafíos de tipo liga de fantasía puedan resolverse
en los demás administradores.
Las 3 tipos utilizan buses de publisher/suscriber.
Tanto la lógica del streaming de partidos reales como los datos de las ligas realesutilizados para resolver los desafíos es la siguiente: Como un usuario puede elegir equipos y armar desafíos de ligas de cualquier parte del mundo, independientemente de dónde viva, entonces cada nivel debe poder su suscribirse tanto al nivel superior como al inferior. Los niveles superiores se utilizan como intermediarios para alcanzar nodos en otras ramas.

Ejemplo: Rodrigo de Tucumán creo un desfafío de basquet utilizando la NBA como liga. Por lo tanto, le gustaría por un
lado poder ver los partidos de la NBA para ir siguiendo sus resultados. Al mismo tiempo, el componente administrador
de Tucumán deberá comenzar a recibir los datos de los partidos de la NBA para calcular los resultados. Estos datos
son provistos por las empresas proveedoras al componente administrador de nivel 3, país=EE.UU.
Entonces el flujo de la suscripción funciona de la siguiente forma. El \textbf{bus del administrador de flujo de streaming} del componente administrador de nivel 4 (provincia = Tucumán)
desea subscribirse al streaming del partido de Spurs vs LA Lakers. Como actualmente no está streameando dicho partido,
se comunica y suscribe all bus de administrador del flujo de streaming de nivel 3 (país=Argentina). Como este tampoco está streameando, se comunica y suscribe al bus del nivel 2 (continente=Sudamérica). Como este tampoco está streameando, se
comunica con el  nivel 1 (Global). Este utiliza el identificador del partido (que contiene el GeoPath) e identifica que el evento es de América del Norte, por lo tanto
se suscribe a dicho partido en el bus de América del Norte. El servidor de América del Norte identifica que el partido se
está jugando en EEUU, por lo que se suscribe al mismo en el bus de dicho país. Finalmente, al llegar el pedido el componente
administrador de EEUU se suscribe al proveedor de transmisiones de la NBA y comienza a recibir la transmisión del partido.
En el proceso se armó la cadena de suscripciones necesaria para recibir los datos. Si algún otro usuario de Argentina o Sudamérica quisiera comenzar a recibir la transmisión también el proceso no tiene necesidad de repetirse.
La lógica de los suscribers y los publishers es parte del conector. De esta manera podemos reutilizar la lógica en cada
nivel, sin necesidad de repetir los componentes.

Finalmente, los datos de la simulación funcionan de forma vertical únicamente, tal como los desafíos. Usuarios de distintos países
o continentes no tienen la necesidad de ver simulaciones compartidas. Por lo tanto, cada componente administrador
se suscribe a lo sumo al nivel i-1 (en caso de que la simulación no esté siendo generada por él mismo), y los paquetes
se propagan de jerarquía mayor a menor dentro de una misma rama.

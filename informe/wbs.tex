\textbf{CU \#7: Posicionándose en ranking jerárquico}
\begin{enumerate}
  \item \textbf{CU\#7-T01}: Setting del entorno de versión actual de 'Curry Game'. Analizar si la tecnología actual puede responder a escalabilidad y performance. (24 horas)
  \item \textbf{CU\#7-T02}: Investigación de distintos tipos de tecnologías si T01 fue negativa. (48 horas)
  lanzar el producto.
  \item \textbf{CU\#7-T03}: Análisis demográfico (cantidad de potenciales usuarios) y disponibilidad de recursos informáticos sobre las regiones, países y continentes en donde se desea
  \begin{enumerate}
    \item \textbf{\textbf{CU\#7-T03-st1}:} Análisis de tecnologías más populares por región (56 horas)
    \item \textbf{\textbf{CU\#7-T03-st2}:} Análisis de tecnologías menos costosas por región (56 horas)
    \item \textbf{\textbf{CU\#7-T03-st3}:} Análisis de compatibilidad de tecnologías entre distintas regiones (56 horas)
  \end{enumerate}
  \item \textbf{CU\#7-T04}: Propuesta y análisis entre distintas arquitecturas (32 horas)
  \item \textbf{CU\#7-T05}: Elección de arquitectura (Tarea hito. 0 horas)
  \item \textbf{CU\#7-T06}: Establecer grafo de conectividad y jerarquía entre servidores (32 horas)
  \item \textbf{CU\#7-T07}: Análisis de estrategias para garantizar disponibilidad y tolerancia a fallas en regiones, países y continentes (40 horas)
  \item \textbf{CU\#7-T08}: Definir conveniencia entre compra y mantenimiento de servidores vs comprar servicios externos (56 horas)
  \item \textbf{CU\#7-T09}: Deploy a primeros servidores y primeras pruebas piloto regionales (50 horas)
  \item \textbf{CU\#7-T10}: Prueba piloto integración de servidores para múltiples regiones (50 horas)
\end{enumerate}


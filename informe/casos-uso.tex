\begin{enumerate}
  \item Simulando desafío de basket según nuevas reglas
  \item Simulando desafíos de otros deportes
  \item Estableciendo estadísticas de jugadores de sitios oficiales y modificadas según redes sociales
  \item Participando en modo liga de fantasía. Usuario. 
  \item Participando en desafío o torneo grupal. Usuario.
  \item Chateando con otros participantes. Usuario.
  \item Posicionándose en ranking jerárquico. Usuario.
  \item Apostando dinero real en desafíos. Usuario.
  \item Controlando acceso de usuarios según leyes de la región respecto a apuestas online
  \item Mirando desafío ficticio en tiempo real. Usuario.
  \item Mirando en vivo partidos de ligas reales. Usuario.
  \item Agregando publicidad en el sitio
  \item Consultando datos estadísticos de usuarios de la aplicación
\end{enumerate}

\subsection{Descripción de los casos de uso, agrupados por área}

\subsubsection{Simulación}

\textbf{Simulando desafío de basket según las nuevas reglas}

Se refiere a modificar la simulación de forma tal que su duración sea similar a la de un partido real. Además incluye incorporar 
nuevas acciones, eventos y características del entorno que modifican la simulación: fouls, tiros libres, cambio de jugadores, cansancio
por minutos en cancha, estadios locales y visitantes, condiciones climáticas, movimientos de los jugadores y posiciones en cancha.

~

\textbf{Simulando desafíos de otros deportes}

Se refiere a la posibilidad de simular desafíos para diversos deportes

~

\textbf{Estableciendo estadísticas de jugadores de sitios oficiales y modificadas según redes sociales}

Se refiere a levantar las estadísticas de jugadores desde sitios oficiales y autorizados. Además las estadísticas
de los jugadores deberán modificarse en función de las menciones de los jugadores hechas en diversas redes sociales,
incluído el sistema de mensajería entre participantes.

\subsubsection{Desafíos}

\textbf{Participando en modo liga de fantasía. Usuario}

Se refiere a agregarle a los usuarios un modo en donde el resultado de los desafíos se definan según el desempeño de jugadores en partidos 
de ligas reales.

~

\textbf{Participando en desafío o torneo grupal. Usuario}

Se refiere a extender la modalidad de los desafíos de forma tal que puedan ser aceptados por varios jugadores. Estos desafíos corresponden a
un único partido, o varios de ellos. Al mismo tiempo pueden ser simulados o resolverse en el modo liga de fantasía.
Finalemnte, extender los desafíos de forma tal que también puedan ser creados por administradores del juego. 

~

\textbf{Chateando con otros participantes. Usuario}

Se refiere a agregar un servicio de mensajería que permita intercambiar mensajes tanto para participantes de un mismo desafío como para
amigos a través de la plataforma.

~

\textbf{Posicionándose en ranking jerárquico. Usuario}

Se refiere a posicionar a los jugadores en ranking jerárquicos: regionales, país, continente y mundial.
De esta manera los jugadores solo pueden acceder a desafíos acordes a su ranking. Este último se resetea cada año.

\subsubsection{Apuestas con dinero real}

\textbf{Apostando dinero real en desafíos. Usuario}

Se refiere a reemplazar el sistema de fichas ficticias por una suma de dinero real al momento de repartir premios por ganar desafíos.
Incluye agregar e integrar el sistema de pagos y cobros, y el ingreso de datos de tarjetas de crédito y cuentas bancarias al sistema.

~

\textbf{Controlando acceso de usuarios según leyes de la región respecto a apuestas online}

Se refiere a evitar el acceso de los usuarios al juego en países o regiones en donde las apuestas en los sitios de internet sean consideradas
ilegales, o bien permitir el registro de usuarios para que participen únicamente de los desafíos gratuitos.



\subsubsection{Streaming}

\textbf{Mirando desafío ficticio en tiempo real. Usuario}

Se refiere a que los usuarios puedan ver el minuto a minuto de los desafíos de forma gráfica. 
Subtareas: Elegir rendering 2D, 3D. Determinar rendering en función de la calidad óptima

~

\textbf{Mirando en vivo partidos de ligas reales. Usuario}

Se refiere a que los usuarios puedan ver desde la app o el sitio partidos de las ligas reales transmitidos en vivo.

\subsubsection{Publicidad y datos estadísticos}

\textbf{Agregando publicidad en el sitio}

Se refiere a desarrollar un mecanismo para que los sponsors e inversores de la aplicación puedan agregar publicidad de forma cómoda.

~

\textbf{Consultando datos estadísticos de usuarios de la aplicación}

Se refiere a la consulta de datos de los usuarios estadísticos por parte de inversores y sponsors para ser utilizados en futuras campañas publicitarias,
de marketing, etc. Los datos recolectados son de caracter demográfico, valor de las apuestas realizadas, jugador más seleccionado en los equipos, etc.

\subsection{Tareas}

\begin{itemize}
  \item 1- Distribuír servidores en distintas regiones. Tiene que ver con arquitectura?
  \item 1- Evaluar disponibilidad de servidores en distintas regiones
  \item 2- Evaluar disponibilidad de servidores en distintas regiones
  \item Ranking jerarquico- Implica análisis de arquitectura
\end{itemize}

\subsection{Análisis de riesgo}

\begin{itemize}
  \item 2- Adaptar el diseño de la versión previa de 'Curry game' para que soporte las nuevas funcionalidades
  \item 3- Adaptar los casos de basket para alguno de los demás deportes puede traer algún contratiempo.
  \item Dinero real- Datos muy sensibles
  \item Ranking jerarquico- Implica análisis de arquitectura
\end{itemize}

\textbf{Controlando acceso de usuarios según leyes de la región}

Se refiere a evitar el acceso de los usuarios al juego en países o regiones en donde las apuestas en los sitios de internet sean consideradas
ilegales, o bien permitir el registro de usuarios para que participen únicamente de los desafíos gratuitos.

~


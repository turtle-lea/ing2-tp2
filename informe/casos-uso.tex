A continuación presentamos el diagrama de casos de uso simplificado (sólo mostramos relaciones entre casos de uso y actores, sin especificar dependencias).

\begin{figure}[h!]
  \centering
  \includegraphics[width=\textwidth]{imagenes/casosDeUso.png}
  \caption{Diagrama de Casos de Uso}
\end{figure}

\newpage

\subsection{Descripción de los casos de uso}

\subsubsection{Cuenta de usuario}
\usecase{01}
{Registrándose en la aplicación. (Participante)}
{Registro de participantes en la aplicación donde les solicitamos sus datos. Esto es necesario para poder autenticarse luego y poder usar la aplicación.}

\usecase{02}
{Autenticándose en la aplicación. (Participante, Administrador)}
{Único medio de ingreso a la aplicación para participar en desafíos o para realizar tareas administrativas (en el caso del Administrador).}

\usecase{03}
{Accediendo a estado de cuenta de participante. (Participante, Administrador)}
{Consulta/modificación de datos personales, dinero ganado, dinero perdido, estadísticas de desafíos ganados, equipos armados, etc. Un Administrador puede acceder con los mismos privilegios que el Participante a cuentas de participantes (como help desk o para extraer datos útiles).}

\subsubsection{Desafíos modo Simulación}

\usecase{04}
{Creando un desafío modo simulación. (Participante, Administrador)}
{Elección del tipo de torneo (plaoffs, liga, combinado zonas, etc), duración, deporte, cantidad máxima de participantes, fecha de inicio. Según la región del participante y según tenga acceso a niveles superiores, podrá elegir una región o se tomará la región del participante para mostrar el desafío a otros participantes de la misma región. Una vez creado empieza la cuenta regresiva para su comienzo. Usa el armado de equipo (CU\#10). Se indica premio si corresponde.}

\usecase{05}
{Participando en desafío modo simulación. (Participante)}
{Elección de un desafío de un listado de desafíos de la región. Usa el pago de la cuota de entrada (CU\#16). Usa también el armado de equipo (CU\#10). Al finalizar el desafío, ganará premios si corresponde.}

\usecase{06}
{Mirando partido simulado en tiempo real. (Participante)}
{Motor de simulación según deporte. Generación de un stream que se traduce a video o se reproduce usando el motor 3D o 2D según se determine en tiempo de ejecución. En cuanto al simulador de basquet, habrá que extenderlo según nuevos requerimientos. El resto de los motores se crean desde cero.}

\subsubsection{Desafíos modo Liga de Fantasía}

\usecase{07}
{Creando desafío modo liga de fantasía. (Participante, Administrador)}
{Elección de la cantidad de fechas reales que abarca, deporte, cantidad máxima de participantes, fecha de inicio. Según la región del participante y según tenga acceso a niveles superiores, podrá elegir una región o se tomará la región del participante para mostrar el desafío a otros participantes de la misma región. Una vez creado empieza la cuenta regresiva para su comienzo. Usa el armado de equipo (CU\#10).}

\usecase{08}
{Participando en desafío modo liga de fantasía. (Participante)}
{Elección de un desafío de un listado de desafíos de la región. Usa el pago de la cuota de entrada (CU\#16). Usa también el armado de equipo (CU\#10). Al finalizar el desafío, ganará premios si corresponde. Los resultados de cada fecha se calculan en base a las estadísticas reales.}

\usecase{09}
{Mirando en vivo partidos de ligas reales. (Participante)}
{Transmisión provista por dueños de los derechos de televisación en vivo de partidos reales.}


\subsubsection{Desafíos (simulados o liga de fantasía)}

\usecase{10}
{Armando equipo. (Participante)}
{Elección de jugadores en cada posición de la cancha y de los suplentes (difiere según deporte del desafío). Elección de un técnico para tener una estrategia de jugada o para sumar puntos según estadísticas (en el caso de liga de fantasía).}

\usecase{11}
{Consultado estado actual de un desafío. (Participante)}
{Estado del partido actual tomado de la simulación o tomado de las estadísticas en tiempo real del partido real. Estado del torneo en general, usando resultados de las simulaciones o los resultados de fechas anteriores de la liga de fantasía. Incluye ranking de participantes.}

\subsubsection{Chat}

\usecase{12}
{Chateando con participantes del desafío. (Participante)}
{Servicio de mensajería que permita intercambiar mensajes entre participantes de un mismo desafío.}

\usecase{13}
{Enviando IM a otro participante. (Participante)}
{Servicio de mensajería instantánea entre dos participantes.}

\subsubsection{Pagos y Cobros}

\usecase{14}
{Ingresando datos de tarjeta de crédito. (Participante)}
{La primera vez que se paga una cuota de entrada a desafío se solicitan los datos de tarjeta de crédito o cuenta corriente bancaria. En el caso de tarjeta de crédito, los datos se almacenan de forma segura y siempre pueden ser modificados.}

\usecase{15}
{Ingresando datos de cuenta corriente. (Participante)}
{La primera vez que se paga una cuota de entrada a desafío se solicitan los datos de tarjeta de crédito o cuenta corriente bancaria. En el caso de cuenta corriente, los datos se almacenan de forma segura y siempre pueden ser modificados.}

\usecase{16}
{Pagando cuota de entrada a desafío. (Participante)}
{Confirmación de datos de pago y confirmación del pago. La comunicación con los sistemas de operaciones bancarias debe ser segura.}

\usecase{17}
{Otorgando créditos a participantes. (Administrador)}
{Les permite ingresar a desafíos sin pagar la cuota. Si ganan, se les resta la cuota del premio.}

\subsubsection{Regionalización}

\usecase{18}
{Consultando un ranking regional. (Participante, Administrador)}
{Rankings de cada nivel. Según el nivel del participante, podrá acceder a uno o más rankings. Siempre puede acceder al de su región. Puede ver estadísticas de todos los participantes y los equipos que fueron usando en los desafíos.}

\usecase{19}
{Accediendo a balances del sitio por región. (Administrador)}
{Información de cuenta y facturación del sitio en cada región, con distintos niveles de granularidad.}

\usecase{20}
{Especificando controles de acceso de usuarios según leyes de la región. (Administrador)}
{Se puede filtrar el acceso total o permitir sólo el acceso a ciertos desafíos: sólo los que son gratuitos y sin premio en dinero, sólo los que son simulados, sólo ciertos deportes, etc. También se puede filtrar acceso a usuarios específicos.}

\subsubsection{Publicidad}

\usecase{21}
{Agregando publicidad en el sitio. (Dueño de Derechos de TV, Representante de Engines Gráficos)}
{ABM para agregar publicidad en los videos de simulación, en el engine 3D/2D, en los videos de partidos reales, o en distintos lugares estratégicos del sitio (a determinar). La interfaz es muy importante por cuestiones de usabilidad de los interesados. El acceso a este ABM está limitado a ciertas IPs y requiere datos de autenticación especiales.}


\subsubsection{Minería de Datos}

\usecase{22}
{Consultando datos estadísticos históricos de usuarios. (Dueño de Derechos de TV, Representante de Engines Gráficos)}
{Se refiere a la consulta de datos estadísticos de los usuarios por parte de inversores y sponsors para ser utilizados en futuras campañas publicitarias, de marketing, etc. Los datos recolectados son de caracter demográfico, valor de las apuestas realizadas, jugador más seleccionado en los equipos, etc.
Todos los datos posibles se pueden consultar y descargar usando datos de autenticación especiales. Se puede filtrar por año, por región, por deporte, etc.}

\usecase{23}
{Consultando datos de preferencia / comportamiento de usuarios. (Dueño de Derechos de TV)}
{Todos los datos posibles se pueden consultar y descargar, usando datos de autenticación especiales. Se pueden obtener usuarios específicos o filtrarlos por deporte, por región, por posición en aĺgún ranking, etc.}
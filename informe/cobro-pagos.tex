\begin{figure}[H]
  \centering
  \includegraphics[width=\textwidth]{imagenes/subs-cobro-y-pago.png}
  \caption{Subsistema de cobros y pagos.}
\end{figure}

Los datos de tarjetas y cuentas y el Tokenizador corren en una maquina distinta a todo el resto. Esto se hace para aislarla lo maximo posible y evitar cualquier vulnerabilidad que pueda llegar a tener el resto del sistema. Ademas estas maquinas deberan tener seguridad fisica, para evitar posibles robos y/o violaciones fisicas al sistema.

Funcionamiento:
\begin{enumerate}
\item {
  \begin{itemize}
  \item Caso medio de pago nuevo:
  Al manejador le llegan pedidos de pagos y/o cobros, autenticados y con informacion de un medio de pago nuevo. Entonces le pasamos esta informacion a el Tokenizador, que le genera un token, persiste en base de datos la informacion y el token correspondiente y le devuelve el token al Manejador de pedidos.
  \item Caso medio de pago existente:
  Al manejador le llegan pedidos de pagos o cobros, autenticados y con un id o refencia minima (elegida en por el Cliente) de que medio de pago se usara. Entonces con ese id o referencia, le pedimos al Tokenizador y obtenemos el token correspondiente.
  \end{itemize}
}

\item Luego con el token correspondiente, se lo brindamos al Realizador de pedidos, que entiende de tokens y con el se comunica con el medio de pago correspondiente y realiza la accion pertinente brindandole el token al medio de pago.

\item Luego con el resultado de la accion, la logeamos en el registro de movimiento y le devolvemos el resultado de la accion al Realizador y este al Manejador
\end{enumerate}

Ademas la información mas sensible (Numero de Tarjeta o Cuenta completa, codigo de seguridad de la tarjeta, etc) seran almacenada encriptada, mediante un algoritmo de clave asimetrica, donde la clave para desencriptar la tengan unicamente los dueños del sistema.

De esta forma, cumpliriamos con un estandar de seguridad llamado PCI, el cual es necesario para para poder realizar cobros y pagos con tarjetas de credito y cuentas bancarias. Este estandar sera todo el tiempo testeado para verificar que estemos siempre cumpliendo y en norma, ya que en caso de no estarlo estariamos abierto a posibles ataques y/o robos de datos.

\subsubsection{Biografia consultada}
\begin{itemize}
\item \href{https://www.pcisecuritystandards.org/documents/Tokenization_Guidelines_Info_Supplement.pdf}{Norma PCI.}\\
\item \href{https://www.quora.com/Do-companies-like-Amazon-etc-have-a-server-farm-to-store-creditcard-information-on-database}{Como cuidan sus datos compañias del estilo Amazon.}
\end{itemize}

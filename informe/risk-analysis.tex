\subsection{Riesgos de casos de uso}

~

\subsubsection{Riesgos rojo}

\noindent\textbf{R\#01: Robo de datos de tarjetas de crédito y/o bancarios [A] [A]} 
\begin{itemize}
	\item{\textbf{Descripción:} Dado que la aplicación será masiva y tendrá muchos datos de usuarios de todo el mundo, entonces (posiblemente) los datos de tarjetas de crédito y bancarios ingresados por los usuarios sean robados y/o publicados.}
	\item{\textbf{Probabilidad:} Alta, los hackers están muy interesados en esta información (más aún si es una aplicación masiva).}
	\item{\textbf{Impacto:} Alto. Ningún usuario volverá a utilizar la aplicación si se roban datos tan sensibles. Además el robo o la publicación de datos bancarios puede tener consecuencias muy graves tanto económicas como legales para el proyecto.}
	\item{\textbf{Mitigación:} Dedicarle el tiempo necesario (o más) al análisis y la elección de una arquitectura adecuada que permita persistir estos datos utilizando estrictas normas de seguridad. Pedir asesoramiento y trabajar conjuntamente con un equipo de expertos en seguridad informática y encriptación. Desarrollar este módulo lo antes posible para tener tiempo de contratar a terceros que lo auditen y lo prueben. Distribuír el almacenamiento de estos datos para reducir el impacto si uno de los servidores se ve comprometido.}
	\item{\textbf{Contingencia:} Bajar el sistema. Contratar a una empresa para que bloquee y denuncie los sitios que tengan plublicados los datos de nuestros usuarios. Tercerizar la búsqueda del error que permitió el robo de datos y la corrección.}
	\item{\textbf{Casos de uso afectados:} CU\#14, CU\#15, CU\#16}
\end{itemize}

~

\noindent\textbf{R\#02: Interrupción del streaming por falla de red [A] [A] } 
\begin{itemize}
	\item{\textbf{Descripción:} Dado que las fallas de red ocurren habitualmente, entonces (posiblemente) la red falle e interrumpa el streaming de un partido.}
	\item{\textbf{Probabilidad:} Alta.}
	\item{\textbf{Impacto:} Alto, dado que puede implicar la pérdida de usuarios.}
	\item{\textbf{Mitigación:} Construir una red que tenga caminos alternativos y garantice alta disponibilidad. Es muy importante que la arquitectura tenga esto en cuenta para que la caída de un tramo de la red no impacte demasiados usuarios (preferentemente ninguno).}
	\item{\textbf{Contingencia:} Contratar servicios externos como YouTube o Vimeo para realizar streaming de partidos mientras se solucionan problemas de red.}
	\item{\textbf{Casos de uso afectados:} CU\#6, CU\#9}
\end{itemize}

~

\noindent\textbf{R\#03: Streaming de video imposibilitado por falta de ancho de banda [A] [A]} 
\begin{itemize}
	\item{\textbf{Descripción:} Dado que los anchos de banda de las conexiones son muy distintos a lo largo del mundo y que un stream de video consume mucho ancho de banda, entonces (posiblemente) los usuarios no puedan ver los desafíos debido a que el streaming requiere de una cantidad de datos mayor a la soportada por la conexión.}
	\item{\textbf{Probabilidad:} Alta.}
	\item{\textbf{Impacto:} Alto. Los usuarios desean ver los desafíos en la mayor calidad posible, pero estarán muy disconformes si no pueden hacerlo en absoluto. Potencialmente dejarán la aplicación por este problema.}
	\item{\textbf{Mitigación:} Dedicarle el tiempo necesario (o más) al análisis y la elección  de una arquitectura adecuada que permita transmitir muchos datos. Hacer un análisis profundo del estado de conectividad de cada región en donde desea lanzarse el producto. Elegir la resolución en función del análisis realizado. Implementar mecanismos de autoadaptación de bitrate según ancho de banda (considerando posibles congestiones en la red).}
	\item{\textbf{Contingencia:} En las regiones donde se hayan producido las fallas, bajar el bitrate al mínimo para garantizar que puedan ser vistos los partidos (con baja calidad).}
	\item{\textbf{Casos de uso afectados:} }
\end{itemize}

~

\noindent\textbf{R\#04: Hackean y modifican el código de un módulo de simulación [A] [A] } 
\begin{itemize}
	\item{\textbf{Descripción:} Dado que los módulos de simulación determinan un ganador por partido y a fin de cuentas esto se traduce en algún premio o compensación para un participante, entonces (posiblemente) haya hackers que intenten atacarnos y alterar el código fuente de los módulos de simulación para beneficiarse en el juego.}
	\item{\textbf{Probabilidad:} Alta.}
	\item{\textbf{Impacto:} Alto.}
	\item{\textbf{Mitigación:} Usar hash de código fuente y verificarlo con cierta frecuencia para reducir la probabilidad de que esto no sea detectado a tiempo. También debemos configurar correctamente firewall y protocolos de transferencia de datos para evitar alteraciones en las simulaciones.}
	\item{\textbf{Contingencia:} Hacer un nuevo deploy del módulo alterado y desactivar las cuentas de los participantes beneficiados. No les podemos quitar los premios a menos que podamos demostrar culpabilidad según las leyes de la región.}
	\item{\textbf{Casos de uso afectados:} CU\#6}
\end{itemize}

~

\noindent\textbf{R\#05: Pérdida de datos estadísticos históricos de participantes usados para minería de datos [M] [A] } 
\begin{itemize}
	\item{\textbf{Descripción:} Dado que cualquier medio de almacenamiento físico podría dañarse, entonces (posiblemente) perderíamos los datos estadísticos históricos de participantes que nuestros sponsors utilizan para minar datos.}
	\item{\textbf{Probabilidad:} Media.}
	\item{\textbf{Impacto:} Alto. El sistema seguirá funcionado, pero los sponsors perderán dinero y corre riesgo la continuidad del proyecto.}
	\item{\textbf{Mitigación:} Priorizar la redundancia de estos datos porque son críticos para que el proyecto genere mucha ganancia monetaria.}
	\item{\textbf{Contingencia:} Según el nivel de péridida de los datos, evaluar posibles compensaciones volviendo a ejecutar simulaciones o recalculando resultados.}
	\item{\textbf{Casos de uso afectados:} CU\#22, CU\#23}
\end{itemize}

~

\noindent\textbf{R\#06: Pérdida parcial o total del log de facturación [M] [A] } 
\begin{itemize}
	\item{\textbf{Descripción:} Dado que el fisco de las distintas regiones podría exigirnos presentar declaraciones juradas con los montos detallados de facturación y dado que ese detalle se guarda en almacenamiento físico propenso a fallas, entonces (posiblemente) podría perderse parcial o totalmente.}
	\item{\textbf{Probabilidad:} Media.}
	\item{\textbf{Impacto:} Alto. No sólo afectaría legalmente al proyecto en las regiones más estrictas, sino que además los sponsors no podrían acceder al estado de cuenta detallado por región (sólo nos quedaría usar las cuentas bancarias, pero sin el detalle generado por nosotros).}
	\item{\textbf{Mitigación:} Priorizar la redundancia de estos datos y regionalizar el log. Esto bajaría la probabilidad de perder datos (porque deberían perderse todas las copias) pero además reduce el impacto (porque si se pierde, se perderá sólo de una región y no de todo el mundo).}
	\item{\textbf{Contingencia:} Usar datos de entidades bancarias que manejan las cuentas de los sponsors y de facturación propia de la aplicación para reconstruir los datos de facturación (aunque perdamos detalles).}
	\item{\textbf{Casos de uso afectados:} CU\#16}
\end{itemize}

~

\noindent\textbf{R\#07: Mal desempeño del render 3D de la simulación en tiempo real [A] [M] } 
\begin{itemize}
	\item{\textbf{Descripción:} Dado que hay una inmensa variedad de dispositivos que van a ejecutar nuestra aplicación, entonces (posiblemente) en algunos de ellos se produzca un mal desempeño del render 3D de la simulación que impida el seguimiento del partido en tiempo real.}
	\item{\textbf{Probabilidad:} Alta.}
	\item{\textbf{Impacto:} Medio, dado que implica pérdida de los seguidores más exigentes de calidad en todo el segmento de dispositivos que no soportan la simulación.}
	\item{\textbf{Mitigación:} Desarrollar un módulo que detecte la capacidad del dispositivo de utilizar el render 3D, y en caso que no lo soporte utilizar el render 2D.}
	\item{\textbf{Contingencia:} Transmitir el partido con render 2D para que todos puedan verlo y mientras tanto analizar cómo solucionar el problema en futuras transmisiones.}
	\item{\textbf{Casos de uso afectados:} CU\#6}
\end{itemize}

\subsubsection{Riesgos amarillo}

\noindent\textbf{R\#08: Imposibilidad de obtener estadísticas de partido/jugadores reales en tiempo real [B] [A] } 
\begin{itemize}
	\item{\textbf{Descripción:} Dado que las estadísticas para usar en simulaciones y en ligas de fantasía se obtienen de un proveedor externo, entonces (posiblemente) pueda fallar el servidor del proveedor y dejaría de funcionar el juego por no poder obtener los datos estadísticos.}
	\item{\textbf{Probabilidad:} Baja, dado que utilizaremos servicios populares que tienen años de experiencia y funcionamiento sin fallas.}
	\item{\textbf{Impacto:} Alto, ya que dejaría de funcionar cualquier simulación o desafíos de liga de fantasía.}
	\item{\textbf{Mitigación:} En caso de falla al obtener datos, realizar reintentos. Si los reintentos fallan, utilizar los últimos datos recibidios. Al mismo tiempo, disparar alarmas a los responsables para intentar solucionar el problema antes de que impacte el juego de manera irreversible.}
	\item{\textbf{Contingencia:} Si el juego debe dejar de funcionar porque pasó mucho tiempo sin obtener datos estadísticos nuevos, se deberá resolver el problema lo antes posible y mientras tanto indicar a los participantes que el sitio se encuentra en mantenimiento.}
	\item{\textbf{Casos de uso afectados:} CU\#6, CU\#9}
\end{itemize}

~

\noindent\textbf{R\#09: Hackean el servidor del proveedor de datos estadísticos [B] [A] } 
\begin{itemize}
	\item{\textbf{Descripción:} Dado que desconocemos las medidas de seguridad usadas por los proveedores de datos estadísticos de partidos, entonces (posiblemente) lo hackeen y modifiquen todos los datos, impactando nuestras simulaciones y resultados de liga de fantasía.}
	\item{\textbf{Probabilidad:} Baja. Los proveedores tienen mucha trayectoria y experiencia y no hay registros de hackeos.}
	\item{\textbf{Impacto:} Alto. Podrían entregarse muchos premios a las personas equivocadas y potencialmente muchos usuarios dejarán de usar la aplicación.}
	\item{\textbf{Mitigación:} Auditoría de la seguridad de los sistemas del proveedor.}
	\item{\textbf{Contingencia:} Suspender todos los partidos y reversar pagos. En un caso tan extremo, todos los premios deberían desestimarse, así como las cuotas de entrada a desafíos en vigencia deberían devolverse.}
	\item{\textbf{Casos de uso afectados:} CU\#6, CU\#9}
\end{itemize}

~

\noindent\textbf{R\#10: Se produce un corte masivo en la red que impide que la mayoría de los usuarios puedan acceder al juego [B] [A] } 
\begin{itemize}
	\item{\textbf{Descripción:} Dado que el juego se accede via internet y se puede acceder en todo el mundo gracias a nuestra infraestructura de red, entonces (posiblemente) se produzca un corte masivo en la red que impida que la mayoría de los usuarios puedan acceder al juego.}
	\item{\textbf{Probabilidad:} Baja.}
	\item{\textbf{Impacto:} Alto. Se pierde mucho dinero.}
	\item{\textbf{Mitigación:} Garantizar alta disponibilidad de nuestro sistema en todos los aspectos funcionales.}
	\item{\textbf{Contingencia:} Actuar lo más rápido posible. Contratar temporalmente servidores de Google y Amazon para volver a poner el juego online.}
	\item{\textbf{Casos de uso afectados:} Todos los casos de uso relacionados al participante.}
\end{itemize}

~

\noindent\textbf{R\#11: Rediseño de la versión anterior del simulador de basquet [M] [M]}
\begin{itemize}
	\item{\textbf{Descripción:} Dado que la versión anterior del simulador puede no ser lo suficientemente extensible/modificable como para poder incorporar los cambios requeridos, entonces (posiblemente) haya que realizar un profundo rediseño y se atrase el proyecto.}
	\item{\textbf{Probabilidad:} Media}
	\item{\textbf{Impacto:} Medio, ya que retrasaría la fecha del release del producto. Los usuarios quieren poder ver estos detalles y nuevas acciones en las simulaciones.}
	\item{\textbf{Mitigación:} Evaluar extensibilidad del diseño para los nuevos requerimientos en el momento de la estimación y asignar la realización de este módulo a la primer etapa de la construcción para evitar posibles retrasos en la fecha de entrega (No se incluye en la etapa de la elaboración ya que está relacionado con la funcionalidad y no con la arquitectura)}
	\item{\textbf{Contingencia:} Hacer el rediseño pensando en funcionalidad y no tanto en un buen diseño, priorizando cumplir con los tiempos.}
	\item{\textbf{Casos de uso afectados:} CU\#6}
\end{itemize}

~

\noindent\textbf{R\#12: Cambia un dato estadístico en tiempo real por un error (ej: se anula un gol) y se paga por error a un participante en un desafío de liga de fantasía [M] [M] } 
\begin{itemize}
	\item{\textbf{Descripción:} Dado que los datos usados para resolver jugadas en los simuladores provienen de proveedores externos en tiempo real, entonces (posiblemente) podría cambiar un dato (ej: se anula un gol) y en la liga de fantasía podría estar pagándose un premio por error.}
	\item{\textbf{Probabilidad:} Media.}
	\item{\textbf{Impacto:} Medio. Esto implicaría una pérdida de dinero ya que no se puede pedir a un participante que devuelva el premio.}
	\item{\textbf{Mitigación:} Antes de pagar premios dejar pasar un tiempo prudencial (de 10 minutos a 1 hora) para que los datos tengan más confiabilidad y luego sí utilizarlos.}
	\item{\textbf{Contingencia:} Suspender el pago de premios de liga de fantasía y agregar nuevas validaciones. Contactar al proveedor y solicitar reducción de frecuencia de fallas.}
	\item{\textbf{Casos de uso afectados:} }
\end{itemize}

~

\noindent\textbf{R\#13: Contratiempos en el desarrollo de simuladores de otros deportes [M] [M]}
\begin{itemize}
	\item{\textbf{Descripción:} Dado que el desarrollo de un simulador para otro deporte (que no sea basquet) puede requerir contemplar situaciones que no han surgido hasta el momento (dado que el dominio es desconocido para los desarrolladores), entonces (posiblemente) surjan contratiempos en el desarrollo de los nuevos simuladores y se atrase el proyecto.}
	\item{\textbf{Probabilidad:} Media}
	\item{\textbf{Impacto:} Medio, ya que retrasaría la fecha del release del producto}
	\item{\textbf{Mitigación:} Si bien resulta razonable tomar por cota superior el tiempo que ha tomado el desarrollo del simulador de basquet para estimar el tiempo de los demás simuladores, estimar un adicional del 20\% del tiempo para cualquier eventualidad}
	\item{\textbf{Contingencia:} Sacar un primer release con menos features de los simuladores para cumplir con la entrega y planificar las mejoras con más tiempo.}
	\item{\textbf{Casos de uso afectados:} CU\#6}
\end{itemize}

~

\noindent\textbf{R\#14: Falla un engine 3D o 2D [M] [M] } 
\begin{itemize}
	\item{\textbf{Descripción:} Dado que los engines 3D y 2D son desarrollados por un proveedor tercerizado, entonces (posiblemente) el software tenga errores y falle en algunos o todos los dispositivos bajo ciertas circunstancias.}
	\item{\textbf{Probabilidad:} Media.}
	\item{\textbf{Impacto:} Medio. La simulación no va a fallar, sólo la renderización, pero esto podría provocar que los usuarios dejen de usar la aplicación porque no les anda.}
	\item{\textbf{Mitigación:} Tercerizar el testing y análisis de los renderizadores 3D y 2D.}
	\item{\textbf{Contingencia:} Utilizar streaming de video en todos los casos (deshabilitar el uso de los engines), hasta que el proveedor corrija el problema.}
	\item{\textbf{Casos de uso afectados:} CU\#6}
\end{itemize}

~

\noindent\textbf{R\#15: Funcionar de forma ilegal en una región [M] [M] } 
\begin{itemize}
	\item{\textbf{Descripción:} Dado que en algunas regiones hay restricciones con respecto a juegos/apuestas online, podría pasar que el sistema funcione de forma ilegal en una región.}
	\item{\textbf{Probabilidad:} Media.}
	\item{\textbf{Impacto:} Medio. Pueden generarse inconvenientes económicos y legales pero la aplicación no dejará de funcionar en las demás regiones donde sí es legal.}
	\item{\textbf{Mitigación:} Realizar pruebas de filtrado de IP en conjunto con empresas de seguridad informática. Desarrollar un módulo de detección de accesos vía proxies y VPNs y bloquearlos. Realizar una reunión con dpto. de seguridad informática de Netflix (que ya tuvieron este problema).}
	\item{\textbf{Contingencia:} Detectar cómo se logró el acceso al sitio (o a funcionalidades del mismo) desde una región no permitida y desarrollar módulos de defensa de ese punto débil.}
	\item{\textbf{Casos de uso afectados:} CU\#1, CU\#2, CU\#3, CU\#18, CU\#20}
\end{itemize}

\subsubsection{Riesgos verde}

\noindent\textbf{R\#16: Mal desempeño del render 2D de la simulación en tiempo real [B] [M] } 
\begin{itemize}
	\item{\textbf{Descripción:} Dado que hay una inmensa variedad de dispositivos que van a ejecutar nuestra aplicación, entonces (posiblemente) en algunos de ellos se produzca un mal desempeño del render 2D de la simulación que impida el seguimiento del partido en tiempo real.}
	\item{\textbf{Probabilidad:} Baja.}
	\item{\textbf{Impacto:} Medio, dado que implica pérdida de los seguidores más exigentes de calidad en todo el segmento de dispositivos que no soportan la simulación.}
	\item{\textbf{Mitigación:} Desarrollar un módulo que detecte la capacidad del dispositivo de utilizar el render 2D, y en caso que no lo soporte utilizar streaming de video.}
	\item{\textbf{Contingencia:} Transmitir el partido como stream de video para que todos puedan verlo y mientras tanto analizar cómo solucionar el problema en futuras transmisiones.}
	\item{\textbf{Casos de uso afectados:} CU\#6}
\end{itemize}

~

\noindent\textbf{R\#17: Se cae la transmisión en vivo de partidos reales [B] [M]} 
\begin{itemize}
	\item{\textbf{Descripción:} Dado que dependemos de un proveedor externo para transmitir partidos reales en vivo, entonces (posiblemente) ocurra una falla y se caiga la transmisión.}
	\item{\textbf{Probabilidad:} Baja.}
	\item{\textbf{Impacto:} Medio. Los más fanáticos se van a enojar mucho si se pierden una anotación de su equipo.}
	\item{\textbf{Mitigación:} Firmar un contrato con las empresas de televisación de partidos que nos garantice 99,99\% de disponibilidad de sus transmisiones.}
	\item{\textbf{Contingencia:} Ofrecer un video del partido a los usuarios afectados para que puedan volver a ver el partido sin cortes.}
	\item{\textbf{Casos de uso afectados:} CU\#9}
\end{itemize}

~

\noindent\textbf{R\#18: Cambia un dato estadístico en tiempo real por un error (ej: se anula un gol) y la simulación utiliza información errónea para determinar resultados de jugadas [M] [B] } 
\begin{itemize}
	\item{\textbf{Descripción:} Dado que los datos usados para resolver jugadas en los simuladores provienen de proveedores externos en tiempo real, entonces (posiblemente) podría cambiar un dato (ej: se anula un gol) y la simulación quizás ya había usado la versión anterior del dato (que era errónea).}
	\item{\textbf{Probabilidad:} Media.}
	\item{\textbf{Impacto:} Bajo. Sólo se verán afectadas ciertas jugadas, que no deberían modificar demasiado el resultado final del partido.}
	\item{\textbf{Mitigación:} Firmar un contrato de calidad de datos con los proveedores, que garantice que el 99,99\% de las consultas devuelvan datos sin errores o previamente validados en los casos más críticos, aunque esto represente una demora en la obtención.}
	\item{\textbf{Contingencia:} Si la diferencia del resultado es muy notoria, anular el partido. Si se repiten estas situaciones, evaluar una estrategia de mediación entre los participantes del partido cuyo resultado se vio afectado para resolver el problema.}
	\item{\textbf{Casos de uso afectados:} CU\#6, CU\#9}
\end{itemize}

~

\noindent\textbf{R\#19: Un simulador no es aprobado en una región por sospecha de fraude [B] [B] } 
\begin{itemize}
	\item{\textbf{Descripción:} Dado que un auditor de cada país evaluará nuestros simuladores para verificar que no haya fraude, entonces (posiblemente) no nos aprueben uno de los simuladores porque se considera fraudulento (aunque no necesariamente lo sea).}
	\item{\textbf{Probabilidad:} Baja.}
	\item{\textbf{Impacto:} Bajo. Los fanáticos de ese deporte no podrán simular partidos. Perderíamos algunos clientes.}
	\item{\textbf{Mitigación:} Enviar gente capacitada junto con el código fuente para explicar detalles de funcionamiento frente a los auditores y así evitar falsas interpretaciones del código. En el peor caso, esta gente puede persuadir al auditor...}
	\item{\textbf{Contingencia:} Se buscará bloquear ciertas partes del funcionamiento o crear un simulador especial para la región si la cantidad de clientes es considerable (esta cantidad se puede determinar en base al uso de la liga de fantasía de ese deporte en esa misma región).}
	\item{\textbf{Casos de uso afectados:} CU\#6}
\end{itemize}












% \textbf{Caso de uso \#1:} Simulando desafío de basket según nuevas reglas
% \begin{itemize}
% \item{\textbf{Riesgo:} La versión anterior del simulador puede no ser lo suficientemente extensible/modificable como para poder incorporar los cambios requeridos, lo cual
% puede implicar un profundo rediseño}
% \item{\textbf{Contingencia:} Asignar la realización de este caso de uso a la primer etapa de la construcción para evitar posibles retrasos en la fecha de entrega. (No se incluye
% en la etapa de la elaboración ya que está relacionado con la funcionalidad y no con la arquitectura)}
% \item{\textbf{Probabilidad:} Media}
% \item{\textbf{Impacto:} Medio, ya que retrasaría la fecha del release del producto. Los usuarios quieren poder ver estos detalles y nuevas acciones en las simulaciones.}
% \end{itemize}

% ~

% \textbf{Caso de uso \#2:} Simulando desafíos de otros deportes
% \begin{itemize}
% \item{\textbf{Riesgo:} Dominio desconocido: Pueden generarse contratiempos en el desarrollo de los simuladores de otro deporte, al ser escencialmente diferentes al basket}
% \item{\textbf{Descripción:} El equipo de desarrollo cuenta con cierta experiencia previa ya que ha implementado el simulador de basket. Sin embargo, el desarrollo de
% un simulador para otro deporte puede requerir contemplar situaciones que no han surgido hasta el momento}
% \item{\textbf{Mitigación:} Si bien resulta razonable tomar por cota superior el tiempo que ha tomado el desarrollo del simulador de basket para estimar el tiempo de los 
% demás simuladores, estimar un adicional del 20\% del tiempo para cualquier eventualidad}
% \item{\textbf{Probabilidad:} Media}
% \item{\textbf{Impacto:} Medio, ya que retrasaría la fecha del release del producto}
% \end{itemize}

% ~

% \textbf{Caso de uso \#7:} Posicionándose en ranking jerárquico. Usuario.
% \begin{itemize}
% \item{\textbf{Riesgo:} La mala elección de la arquitectura o de la distribución de los servidores puede degradar gravemente la performance. La correcta implementación de este caso de uso es vital para el 'core' de la aplicación, debido a la masiva cantidad de usuarios que tenga el sistema}
% \item{\textbf{Contingencia:} Asignarle al desarrollo de este caso de uso un papel central en el proceso. Dedicarle el tiempo necesario (o más) al análisis y la elección 
% de una arquitectura adecuada, una posible distribución de servidores y la elección de empresas que nos provean de dichos servidores vs la compra de servidores. Se le ha asignado
% un lugar en la 1ra iteración del proceso de elaboración}
% \item{\textbf{Probabilidad:} Alta}
% \item{\textbf{Impacto:} Alto. Una performance degradada a nivel general puede implicar graves riesgos para el producto}
% \end{itemize}

% ~

% \textbf{Caso de uso \#8:} Apostando dinero real en desafíos. Usuario.
% \begin{itemize}
% \item{\textbf{Riesgo:} Los datos de tarjetas de crédito y bancarios ingresados por los usuarios pueden ser robados o publicados. Se está trabajando con data extremadamente
% sensible}
% \item{\textbf{Contingencia:} Asignarle al desarrollo de este caso de uso un papel central en el proceso. Dedicarle el tiempo necesario (o más) al análisis y la elección 
% de una arquitectura adecuada que permita persistir estos datos utilizando estrictas normas de seguridad. Pedir asesoramiento y trabajar conjuntamente con un equipo de expertos en seguridad informática y encriptación. Se le ha asignado un lugar en la 2da iteración del proceso de elaboración}
% \item{\textbf{Probabilidad:} Alta}
% \item{\textbf{Impacto:} Alto. El robo o la publicación de datos bancarios puede tener consecuencias muy graves tanto económicas como legales para el proyecto}
% \end{itemize}

% ~

% \textbf{Caso de uso \#9:} Controlando acceso de usuarios según leyes de la región respecto a apuestas online
% \begin{itemize}
% \item{\textbf{Riesgo:} Actuar de forma ilegal en aquellas regiones en donde las apuestas están prohibidas}
% \item{\textbf{Contingencia:} Reflexionar en conjunto con el caso de uso \textbf{\#8}} 
% \item{\textbf{Probabilidad:} Media}
% \item{\textbf{Impacto:} Medio. Pueden generarse inconvenientes económicos y legales}
% \end{itemize}

% ~

% \textbf{Caso de uso \#10 y \#11:} Streaming de desafíos
% \begin{itemize}
% \item{\textbf{Riesgo:} Los usuarios no pueden ver los desafíos debido a que el streaming requiere de una cantidad de datos mayor a la soportada por la conexión}
% \item{\textbf{Contingencia:} Asignarle al desarrollo de este caso de uso un papel central en el proceso. Dedicarle el tiempo necesario (o más) al análisis y la elección 
% de una arquitectura adecuada que permita transmitir muchos datos. Hacer un análisis profundo del estado de conectividad de cada región en donde desea lanzarse el prodcuto.
% Elegir el rendering en 2D/3D en función del análisis. Para el caso del streaming de los partidos reales, elegir la resolución en función del análisis realizado}
% \item{\textbf{Probabilidad:} Alta}
% \item{\textbf{Impacto:} Medio. Los usuarios desean ver los desafíos en la mayor calidad posible, pero estarán muy disconformes si no pueden hacerlo en absoluto}
% \end{itemize}


% \subsection{Tabla de análisis de riesgos}

% \begin{center}
%     \begin{tabular}{ | l | l | l | p{5cm} |}
%     \hline
%     Impacto | Probabilidad & Alta & Media & Baja \\ \hline
%     Alto & (7)-(8) & - & - \\ \hline
%     Medio & (10)-(11) & (1)-(2)-(9) & - \\ \hline
%     Bajo & - & - & - \\ \hline
%     \end{tabular}
% \end{center}

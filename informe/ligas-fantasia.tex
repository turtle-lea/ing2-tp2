\begin{figure}[H]
  \centering
  \includegraphics[width=\textwidth]{imagenes/fantasia.png}
  \caption{Subsistema de Ligas de Fantasia.}
\end{figure}

El procesador de comienzo de desafío almacena el desafío nuevo en el repositorio de desafíos activos, junto con todos los partidos a tener en cuenta.
Luego crea un procesador de comienzo de partido, que se encarga de revisar si el desafío indicado tiene algún partido que comenzar, y en caso que no sea así, programa un timer para
que le avise cuándo comienza el próximo partido. En este esquema, se crea un procesador de comienzo de partido para cada desafío nuevo que llega.
Cada vez que hay un partido, el procesador de comienzo de partido crea un procesador de minuto a minuto de partido. Este último, cada un minuto,
solicita al Subsistema de Estadisticas de Partidos las últimas actualizaciones. Con ellas evalúa las reglas de puntajes y actualiza los rankings (por ejemplo, una regla podría ser "Si mete gol, suma 3 puntos", y si una actualización
indica que Agüero metió un gol, entonces se envia al ranking A todos los participantes cuyo equipo tenga a Agüero suma 3 puntos).
Una vez actualizado los rankings, vuelve aprogramar el timer por un minuto. De esta manera habrá un procesador minuto a minuto por cada partido activo, que en cada minuto
actualizará los puntajes del ranking.\\

El procesador de comienzo de partido, luego de crear el de minuto a minuto, se programa
para el próximo partido. Si no hay más partidos, simplemente se destruye a sí mismo.\\

Cuando cada partido termina, el procesador de minuto a minuto se destruye a sí mismo, pero antes avisa del fin de partido al procesador de fin de partido, que verifica si
fue el último del desafio y en tal caso informa de la finalización del partido al subsistema de desafíos.

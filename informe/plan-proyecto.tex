\begin{enumerate}
  \item \Large{\textbf{Elaboración}}
  \begin{enumerate}
    \item \large{\textbf{Iteración 0}}
    \begin{enumerate}
      \item \normalsize{\textbf{Reconocimiento de casos de uso}}
      \item \normalsize{\textbf{Priorización de casos de uso}}
      \item \normalsize{\textbf{Análisis de riesgos}}
      \item \normalsize{\textbf{Generar esquema del plan general del proyecto}}
      \item \normalsize{\textbf{Realización de WBS de la primera iteración}}
    \end{enumerate}
    \item \large{\textbf{Iteración 1}}
    \begin{enumerate}
      \item \normalsize{\textbf{Caso de uso \#7:} Posicionándose en ranking jerárquico}
    \end{enumerate}
    \item \large{\textbf{Iteración 2}}
    \begin{enumerate}
      \item \normalsize{\textbf{Caso de uso \#8:} Apostando dinero real en desafíos}
    \end{enumerate}
    \item \large{\textbf{Iteración 3}}
    \begin{enumerate}
      \item \normalsize{\textbf{Caso de uso \#10 y \#11:} Streaming de desafíos}
    \end{enumerate}
  \end{enumerate}
  \item \Large{\textbf{Construcción}}
  \begin{enumerate}
    \item \large{\textbf{Iteración 1: Integración y simulación}}
    \begin{enumerate}
      \item \normalsize{\textbf{Integración:} Integración y completar tareas faltantes respecto a los 3 casos de uso anteriores}
      \item \normalsize{\textbf{Caso de uso \#1:} Simulando desafío de basket según nuevas reglas}
      \item \normalsize{\textbf{Caso de uso \#2:} Simluación de desafíos de otros deportes}
    \end{enumerate}
    \item \large{\textbf{Iteración 2: Desafíos}}
    \begin{enumerate}
      \item \normalsize{\textbf{Caso de uso \#4:} Participando en modo liga de fantasía}
      \item \normalsize{\textbf{Caso de uso \#5:} Participando en desafío o torneo grupal}
      \item \normalsize{\textbf{Caso de uso \#6:} Chateando con otros participantes}
    \end{enumerate}
    \item \large{\textbf{Iteración 3: Publicidad y datos estadísticos}}
    \begin{enumerate}
      \item \normalsize{\textbf{Caso de uso \#12:} Agregando publicidad en el sitio}
      \item \normalsize{\textbf{Caso de uso \#13:} Consultando datos estadísticos de usuarios de la aplicación}
    \end{enumerate}
  \end{enumerate}
  \item \Large{\textbf{Transición}}
  \begin{enumerate}
    \item \normalsize{Tareas de mantenimiento}
  \end{enumerate}
\end{enumerate}

Al momento de la entrega de este documento la iteración \#0 ya ha sido completada. En consecuencia, se incluye en este informe las conclusiones y resultados obtenidos en esta
primera iteración.

En el proceso de elaboración se establece y valida la arquitectura del sistema, y se genera una base ejecutable del código sobre la cual se pueda empezar a testear dicha arquitectura.
Al mismo tiempo se desarrollan los componentes con un rol central dentro de la arquitectura definida. Por este motivo, decidimos incluír los 3 casos de uso que tienen decisiones 
de alto impacto a nivel estructural, dejando cada una en una iteración diferente. La idea es construir la arquitectura de forma incremental. La opción a nivel arquitectura propuesta
en cada iteración debe ser compatible con la anterior, y al mismo tiempo solucionar el caso de uso actual.

En la etapa de construcción se desarrollan los features faltantes de forma iterativa incremental. En la primera iteración reservamos una serie de tareas de integración o implementación
de cosas faltantes, relacionadas con la funcionalidad y no con lo estructural. Luego, se desarrollan los features del sistema faltantes.

Finalmente en la etapa de transición se realizan tareas de mantenimiento.
